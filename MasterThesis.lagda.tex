\documentclass{article}[12pt]
\usepackage[a4paper, left=30mm,
 top=30mm,  right=30mm,
 bottom=20mm]{geometry}
\usepackage[utf8]{inputenc}
\usepackage[english,russian]{babel}
\usepackage[hidelinks]{hyperref}
\usepackage{amsmath}
\usepackage{amsfonts}
\usepackage{hyphenat}
\usepackage{tikz-cd}
\usetikzlibrary{babel}
\usepackage{letltxmacro}
\usepackage{amsthm}
\usepackage{amssymb}
\usepackage{tikz-cd}
\usepackage{proof}
\usepackage{dirtytalk}
\usepackage[links]{agda}
\usepackage{newunicodechar}
\usepackage{microtype}
\usepackage{bussproofs}
\usepackage{multicol}
\usepackage{dsfont}

\DisableLigatures[-]{encoding=T2A}

\newunicodechar{λ}{\ensuremath{\mathnormal\lambda}}
\newunicodechar{←}{\ensuremath{\mathnormal\from}}
\newunicodechar{→}{\ensuremath{\mathnormal\to}}
\newunicodechar{∀}{\ensuremath{\mathnormal\forall}}
\newunicodechar{𝒰}{\ensuremath{\mathcal{U}}}
\newunicodechar{≡}{\ensuremath{\mathnormal{\equiv}}}
\newunicodechar{⊔}{\ensuremath{\mathnormal{\sqcup}}}

\newtheorem{theorem}{Теорема}
\newtheorem{lemma}{Лемма}
\newtheorem{proposition}{Утверждение}
\newtheorem{definition}{Определение}
\newtheorem{corollary}{Следствие}

\newcommand{\dash}{\textemdash\ }
\newcommand{\ind}{\operatorname{ind}}

\title{Гомотопическая теория типов и ее модели}
\author{Глеб Красилич}
\date{Май 2023}

\begin{document}

\maketitle

\section{Теории типов}

Гомотопическая теория типов (Homotopy Type Theory, HoTT) является расширением интуиционистской теории типов
Мартин-Лёфа (Martin-Löf Type Theory, MLTT), которая в свою очередь является выразительным
продолжением простого типизованного $\lambda$-исчисления до формальных оснований конструктивной математики.
Напомним основные факты о $\lambda$-исчислениях.

\subsection{Исчисление $\lambda_\rightarrow$}

Зафиксируем алфавит типовых переменных $TVar = \{p, q, v, \dots \}$ и алфавит $\lambda$-переменных
$\Lambda Var = \{x, y, z, \dots \}$
\begin{definition}
    Язык типов \dash это наименьшее множество $Types$ такое, что
    \begin{enumerate}
        \item $TVar \subset Types$.
        \item Если слова $\phi$ и $\psi$ принадлежат $Types$, то слово $(\phi \rightarrow \psi)$
        также принадлежит $Types$.
    \end{enumerate}
\end{definition}
Иными словами, язык типов можно задать грамматикой
$$type := pvar | (type \rightarrow type)$$
где $pvar \in TVar$. Если считать, что множество типовых переменных $TVar$ совпадает с множеством
пропозициональных переменных логики высказываний, то язык типов $Types$ в точности совпадает с
языком импликативного фрагмента (интуиционистской) логики высказываний $IPC$.  

\begin{definition}
    Язык термов просто типизованного по Чёрчу $\lambda$-исчисления \dash
    это наименьшее множество $\Lambda Terms$ такое, что
    \begin{enumerate}
        \item $\Lambda Var \subset \Lambda Term$.
        \item Если термы (слова) $M$ и $N$ принадлежат $\Lambda Terms$, то терм
        $(MN)$ также лежит в $\Lambda Terms$.
        \item Если $M \in \Lambda Terms$, то $(\lambda x : \phi . M) \in \Lambda Terms$
        для любой переменной $x \in \Lambda Var$ и любого типа $\phi \in Types$.
    \end{enumerate}
\end{definition}
Таким образом грамматика $\lambda$-термов имеет следующий вид:
$$t := x | (tt) | (\lambda x : \phi . t)$$
где $x$ \dash произвольная $\lambda$-переменная, а $\phi$ \dash произвольный тип.
Приведем несколько примеров грамматически корректных $\lambda$-термов:
$$x$$
$$((xx)x)$$
$$(\lambda x : p . x)$$
$$((\lambda x : (p \rightarrow q).y)z)$$

Естественно, скобки, необходимые для однозначности разбора, мы будем опускать для лучшей читаемости:
мы будем считать, что операция $\lambda$-абстракция (термы вида $\lambda x : \phi . M$) имеет более
низкий приоритет, чем $\lambda$-применение (термы вида $MN$), и не будем писать самые внешние скобки.
Мы также считаем, что при чтении терма, составленного с помощью операций равного приоритета,
скобки расставляются лево-ассоциативным образом, то есть, например, терм $xyz$ читается как $((xy)z)$
(в отличии от языка тиров/имликативного фрагмента логики высказываний, где скобки расставляются
право-ассоциативно: $p \rightarrow q \rightarrow v$ читается как $(p \rightarrow (q \rightarrow v))$).

Введем понятие свободных и связанных переменных.
\begin{definition}
    Определим рекурсивно функцию $FVar : \Lambda Terms \rightarrow \mathcal{P}(TVar)$,
    переводящую $\lambda$-терм в его множество так называемых свободных переменных:
    \begin{enumerate}
        \item $FVar(x) = \{ x \}$, если $x$ \dash $\lambda$-переменная.
        \item $FVar(MN) = FVar(M) \cup FVar(N)$.
        \item $FVar(\lambda x : \phi . M) = FVar(M) \setminus \{ x \}$.
    \end{enumerate}
    Если переменная $x$ входит в $\lambda$-терм $M$ как подтерм, но $x \not \in FVar(M)$, то $x$
    называется связанной переменной терма $M$. Терм беза свободных переменных называется
    комбинатором, или просто замкнутым термом.
\end{definition}

Теперь определим на $\lambda$-исчислении структуру формальной системы переписывания термов.
Напомним, что формальная система преобразований \dash это тройка $(A, \rightarrow, \twoheadrightarrow)$,
где $A$ \dash множество-носитель, $\rightarrow$ \dash бинарное отношение на $A$, а
$\twoheadrightarrow$ \dash транзитивно-рефлексивное замыкание $\rightarrow$. Если $A$ является
множеством слов некоторого формального языка, то тройку $(A, \rightarrow, \twoheadrightarrow)$ также
принято называть формальной системой переписывания термов, так как отношение $\rightarrow$
естественно воспринимать как правило преобразования слов.

Во-первых, нам бы хотелось иметь возможность заменять имена связанных переменных. Делается
это с помощью отношения $\alpha$-конверсии $\rightarrow_\alpha$: Если в терме $M$ есть подтерм вида
$\lambda x : \phi . N$, то $M \rightarrow_\alpha M'$, где терм $M'$ получается из $M$ заменой
какого либо единственного вхождения подтерма $\lambda x : \phi . N$ на подтерм $\lambda y : \phi . N[y/x]$, при условии, что
в $y \not \in FVar(N)$ (запись $N[y/x]$ означает, терм $N$ в котором все свободные вхождении переменной
$x$ заменены на переменную $y$). Приведем примеры термов, находящихся в отношении $\alpha$-конверсии:
$$(\lambda x : \phi . x) \rightarrow_\alpha (\lambda y : \phi . y)$$
$$((\lambda x : \phi . y) x) \rightarrow_\alpha ((\lambda z : \phi . y) x)$$
А терм $\lambda x : \phi . y$, например, не находится в отношении $\alpha$-конверсии с термом
$\lambda y : \phi . y$, так свободная переменная $y$ становится связанной. Также как и
терм $(\lambda x_1 : \phi . x_1)(\lambda y_1 : \psi . y_1)$ не находится в отношении $\alpha$-конверсии
с термом $(\lambda x_2 : \phi . x_2)(\lambda y_2 : \psi . y_2)$, так как
здесь происходит переименование переменных в двух подтермах одновременно (однако данные термы
находятся в транзитивном отношении $\twoheadrightarrow_\alpha$).

Во-вторых, мы свяжем $\lambda$-абстракции с функциями от связанных переменных,
а $\lambda$-применения \dash с вычислениями функций на аргументах.
Пусть в терме $M$ есть подтерм $N = (\lambda x : \phi . L)K$. Тогда $M$ 
находится в отношении  $\beta$-редукции $M \rightarrow_\beta M'$ с некоторым термом $M'$, 
получаемым из $M$ заменой какого-либо одного вхождения подтерма $N$ на подтерм
$L[K/x]$ при условии корректности данной подстановки: никакая свободная переменная $K$ не
становится после такой подстановки связанной. Приведем несколько примеров:
$$(\lambda x : \phi . x) y \rightarrow_\beta y$$
$$(\lambda y : \phi . y) y \rightarrow_\beta y$$
$$(\lambda x : \phi . y)((\lambda y : \psi . y) z) \rightarrow_\beta (\lambda x : \phi . y) z \rightarrow_\beta y$$

Таким образом, положив отношение $\rightarrow$ равным объединению $\rightarrow_\alpha$ и $\rightarrow_\beta$,
получаем формальную систему переписывания термов $(\Lambda Terms, \rightarrow_{\alpha \beta}, \twoheadrightarrow_{\alpha \beta})$.

\begin{definition}
    \label{NormalFormDefinition}
    Пусть $(A, \rightarrow, \twoheadrightarrow)$ \dash формальная система переписывания термов.
    Элемент $a \in A$ является нормальной формой, если $\forall b \in A (a \not \rightarrow b)$.
\end{definition}

Общее определение \ref{NormalFormDefinition}, однако, не очень хорошо работает
для нашей системы $(\Lambda Terms, \rightarrow_{\alpha \beta}, \twoheadrightarrow_{\alpha \beta})$:
к любому замкнутому терму $M$ можно применить $\alpha$-конверсию. Поэтому мы будем говорить 
о нормальной форме с точностью до $\alpha$-конверсии: терм $M$ находится в нормальной форме
если $\forall N \in \Lambda Term (M \rightarrow_{\alpha \beta} N \Rightarrow M \rightarrow_\alpha N)$.

\begin{definition}
    \label{NormalizationDefinition}
    Элемент $a$ системы $(A, \rightarrow, \twoheadrightarrow)$ называется (слабо-)нормализуемым, если
    найдется такой $b \in A$, что 
    \begin{enumerate}
        \item $b$ является нормальной формой.
        \item $a \twoheadrightarrow b$.
    \end{enumerate}
    (В частности, всякая нормальная форма является нормализуемой, в силу рефлексивности отношения
    $\twoheadrightarrow$.)
\end{definition}

\begin{definition}
    \label{StrongNormalizationDefinition}
    Элемент $a$ системы $(A, \rightarrow, \twoheadrightarrow)$ называется сильно-нормализуемым,
    если
    \begin{enumerate}
        \item $a$ является нормализуемым.
        \item Не верно следующие утверждение: существует счетная последовательность
        элементов $A$ $\{ b_0, b_1, b_2, \dots \}$ такая, что $b_0 = a$ и 
        $b_i \rightarrow b_{i+1}$ для любого натурального $i$.
    \end{enumerate}
\end{definition}
Иными словами, определение \ref{StrongNormalizationDefinition} говорит, что $a$ является
сильно-нормализуемым, если всякая последовательность преобразований $\rightarrow$ элемента $a$ приводит
к нормальной форме за конечное число шагов.

Проиллюстрируем определения \ref{NormalFormDefinition} и \ref{StrongNormalizationDefinition} примерами из
нашего $\lambda$-исчисления. Терм $(\lambda x : p . y)((\lambda z : q . z) w)$ является сильно нормализуемым:
мы либо можем переписать внутренний подтерм $(\lambda z : q) w$ с помощью $\beta$-редукции и получить
терм $(x : p . y) w$, который единственным образом $\beta$-редуцируется до нормальной формы $y$, либо
либо сразу переписать самое внешние $\lambda$-применение до $y$. Других последовательностей $\beta$-редукций
для такого терма нет. С другой стороны, терм $(\lambda x : p . y)((\lambda z : q . zz)(\lambda z : q . zz))$
является только слабо-нормализуемым: как и в прошлом примере, мы можем сразу средуцироваться до $y$,
или попытаться переписывать подтерм $(\lambda z : q . zz)(\lambda z : q . zz)$ и получить бесконечную последовательность
$$(\lambda x : p . y)((\lambda z : q . zz)(\lambda z : q . zz)) \rightarrow_\beta (\lambda x : p . y)((\lambda z : q . zz)(\lambda z : q . zz)) \rightarrow_\beta \dots$$
Сразу же замечаем, что терм $(\lambda z : q . zz)(\lambda z : q . zz)$ не является ни сильно-,
ни слабо-нормализуемым.

\begin{definition}
    \label{ChurchRosserDefinition}
    Система $(A, \rightarrow, \twoheadrightarrow)$ обладает свойством Чёрча-Россера (свойством конфлюэнтности),
    если 
    $$\forall a, b, b' \in A \Bigl((a \twoheadrightarrow b \land a \twoheadrightarrow b') \Rightarrow \exists c \in A (b \twoheadrightarrow c \land b' \twoheadrightarrow c) \Bigr)$$
    $$
    \begin{tikzcd}
        & \arrow[ld, two heads] a \arrow[rd, two heads] & \\
        b \arrow[rd, two heads, dashed] & & \arrow[ld, two heads, dashed] b' \\
        & c &
    \end{tikzcd}
    $$
\end{definition}

\begin{lemma}
    \label{UniqueNormalFormLemma}
    Если для системы $(A, \rightarrow, \twoheadrightarrow)$ выполняется свойство Чёрча-Россера, и
    $a$ \dash нормализуемый элемент $A$, то его нормальная форма $b$ единственна.
\end{lemma}
\begin{proof}
    Пусть $a$ редуцируется к двум нормальным формам $b$ и $b'$. Тогда по свойству
    конфлюэнтности найдется такой $c$, что $b \twoheadrightarrow c$ и $b' \twoheadrightarrow c$. Но
    любая нормальная форма находится в отношении $\twoheadrightarrow$ только сама с собой.
    А значит
    $$b = c = b'$$
\end{proof}

\begin{theorem}[\cite{SorUrz06}]
    Система $(\Lambda Terms, \rightarrow_{\alpha \beta}, \twoheadrightarrow_{\alpha \beta})$ обладает
    свойством Чёрча-Россера.
\end{theorem}

Существует несколько способов закодировать натуральные числа в $\lambda$-исчислении. Один из таких
способов \dash построить для каждого натурального $n$ комбинатор $[n]$, называемый нумералом Чёрча,
следующим образом:
$$[0] := \lambda f : p \rightarrow p . \lambda x : p . x$$
$$[1] := \lambda f : p \rightarrow p . \lambda x : p . fx$$
$$[2] := \lambda f : p \rightarrow p . \lambda x : p . f (fx)$$
$$[3] := \lambda f : p \rightarrow p . \lambda x : p . f (f (fx))$$
$$ \dots $$
$$[n] := \lambda f : p \rightarrow p. \lambda x : p . \underbrace{f(\dots (f}_{n\ \text{применений}\ f} x))$$
$$ \dots $$

\begin{theorem}[\cite{SorUrz06}]
    Для всякой вычислимой функции $f$ из $\mathbb{N}$ в $\mathbb{N}$ существует
    $\lambda$-терм $F$ такой, что
    \begin{enumerate}
        \item Если $n$ не принадлежит области определения $f$, то терм $F[n]$ не нормализуем.
        \item Если $n$ принадлежит области определения $f$, то $F[n] \twoheadrightarrow_{\alpha \beta} [f(n)]$.
    \end{enumerate}
\end{theorem}

По лемме \ref{UniqueNormalFormLemma} всякий нумерал Чёрча единственен с точностью до переименования
связанных переменных, а значит редукция термов $F[n]$ задаёт корректною функциональное отношение на
$\mathbb{N}$. Таким образом система $(\Lambda Terms, \rightarrow_{\alpha \beta}, \twoheadrightarrow_{\alpha \beta})$
является полноценной, полной по Тьюрингу моделью вычислений.

Однако, до сих в пор в нашем языке $\lambda$-исчисления остаются термы вида $xx$, которым невозможно приписать
какое-либо вычислительный смысл. Мы "отфильтруем" такие "плохие" термы, введя дедуктивную систему о типах:
мы будем делить типы на те, для которых можно вывести суждение о наличие у них некоторого типа (типизуемый терм),
и те, у которых такое суждение вывести нельзя (нетипизуемые термы).

\begin{definition}
    \label{ContextDefinition}
    Контекст $\Gamma$ \dash это конечное, возможно пустое, множество пар вида $x : \phi$, где $x$ \dash $\lambda$-переменная,
    а $\phi$ \dash некоторый тип ($\phi \in Types$), причем для любых двух пар $(a : \phi)$ и $(b : \psi)$ из
    $\Gamma$ верно
    $$a = b \Rightarrow \phi = \psi$$
    Запись $\Gamma, x : \phi$ означает контекст $\Gamma \cup \{x: \phi\}$.
\end{definition}

\begin{definition}
    \label{JudgmentDefinition}
    Суждение \dash это формальное утверждение вида
    $$\Gamma \vdash M : \phi$$
    которое читается как "в контексте $\Gamma$ терм $M$ имеет тип $\phi$".

    Если $\Gamma = \emptyset$, то вместо 
    $$\emptyset \vdash M : \phi$$ 
    пишут просто
    $$\vdash M : \phi$$

    Вывод суждение $\Gamma \vdash M : \phi$ \dash это конечное дерево, чьи вершины помечены суждениями,
    корень помечен суждением $\Gamma \vdash M : \phi$, 
    листья помечены аксиомами (правилами без посылок)
    \begin{prooftree}
        \AxiomC{}
            \RightLabel{$Var$}
        \UnaryInfC{$\Gamma, x : \phi \vdash x : \phi$}
    \end{prooftree}
    а ребрам соответствуют правила вывода, описывающие способы получения одних суждений из других:
    \begin{multicols}{2}
    \begin{prooftree}
        \AxiomC{$\Gamma, x : \phi \vdash M : \psi$}
            \RightLabel{$Abs$}
        \UnaryInfC{$\Gamma \vdash (\lambda x : \phi . M) : \phi \rightarrow \psi$}
    \end{prooftree}
    \begin{prooftree}
        \AxiomC{$\Gamma \vdash M : \phi \rightarrow \psi$}
        \AxiomC{$\Gamma \vdash N : \phi$}
            \RightLabel{$App$}
        \BinaryInfC{$\Gamma \vdash (MN) : \psi$}
    \end{prooftree}
    \end{multicols}
    Во всех правилах вывода, $\Gamma$ \dash произвольный контекст.
\end{definition}

Приведем несколько примеров выводов суждений. Для начала, выведем суждение 
$x : p \rightarrow q, y : p \vdash (xy) : q$
\begin{prooftree}
    \AxiomC{}
        \RightLabel{$Var$}
    \UnaryInfC{$ x : p \rightarrow q, y : p \vdash x : p \rightarrow q$}
    \AxiomC{}
        \RightLabel{$Var$}
    \UnaryInfC{$ x : p \rightarrow q, y : p \vdash y : p$}
        \RightLabel{$App$}
    \BinaryInfC{$ x : p \rightarrow q, y : p \vdash (xy) : q$}
\end{prooftree}
Другой пример: $\vdash (\lambda x : p . x) : p \rightarrow p$
\begin{prooftree}
    \AxiomC{}
        \RightLabel{$Var$}
    \UnaryInfC{$ x : p \vdash x : p$}
        \RightLabel{$Abs$}
    \UnaryInfC{$\vdash (\lambda x : p . x) : p \rightarrow p$}
\end{prooftree}
Ну и третий пример: $\vdash (\lambda f : p \rightarrow p . \lambda x : p . f (f x)) : (p \rightarrow p) \rightarrow p \rightarrow p$
\begin{prooftree}
    \AxiomC{}
        \RightLabel{$Var$}
    \UnaryInfC{$f : p \rightarrow p, x : p \vdash f : p \rightarrow p$}
    \AxiomC{}
        \RightLabel{$Var$}
    \UnaryInfC{$f : p \rightarrow p, x : p \vdash f : p \rightarrow p$}
    \AxiomC{}
        \RightLabel{$Var$}
    \UnaryInfC{$f : p \rightarrow p, x : p \vdash x : p$}
        \RightLabel{$App$}
    \BinaryInfC{$f : p \rightarrow p, x : p \vdash (f x) : p$}
        \RightLabel{$App$}
    \BinaryInfC{$f : p \rightarrow p, x : p \vdash (f (f x)) : p$}
        \RightLabel{$Abs$}
    \UnaryInfC{$f : p \rightarrow p \vdash (\lambda x : p . f (f x)) : p \rightarrow p$}
        \RightLabel{$Abs$}
    \UnaryInfC{$\vdash (\lambda f : p \rightarrow p . \lambda x : p . f (f x)) : (p \rightarrow p) \rightarrow p \rightarrow p$}
\end{prooftree}

Естественно, система переписывание термов $(\Lambda Terms, \rightarrow_{\alpha \beta}, \twoheadrightarrow_{\alpha \beta})$
согласованна с дедуктивной системой вывода суждений о типизации.
\begin{lemma}[\cite{SorUrz06}]
    Пусть $M$ \dash $\lambda$-терм такой, что
    \begin{enumerate}
        \item Суждение $\Gamma \vdash M : \phi$ выводимо для некоторого контекста $\Gamma$
        и некоторого типа $\phi$.
        \item $M \twoheadrightarrow_{\alpha \beta} N$ для некоторого терма $N$.
    \end{enumerate}
    Тогда суждение $\Gamma \vdash N : \phi$ также выводимо.
\end{lemma}

Систему переписываний $(\Lambda Terms, \rightarrow_{\alpha \beta}, \twoheadrightarrow_{\alpha \beta})$
вместе с дедуктивной системой из определения \ref{JudgmentDefinition} принято называть
простым типизованным по Чёрчу $\lambda$-исчислением и обозначать $\lambda_\rightarrow$.

\begin{theorem}
    \label{SimplyTypedStrongNormaliztionTheorem}
    Пусть $M$ \dash типизуемый в некотором контексте $\Gamma$ терм (для него выводится суждение $\Gamma \vdash M : \phi$ для некоторого типа $\phi$).
    Тогда $M$ является сильно-нормализуемым.
\end{theorem}

Следствием теоремы \ref{SimplyTypedStrongNormaliztionTheorem} является тот факт, что типизуемые термы
больше не являются полной по Тьюрингу моделью вычислений. Вместо этого типизуемым термам
соответствует некоторое собственное подмножество тотальных вычислимых функций, называемое расширенными
многочленами \cite{Zakr07}: наименьшее множество вычислимых функций,
замкнутое относительно взятия композиции, содержащие
\begin{enumerate}
    \item Проекции из кортежей натуральных чисел.
    \item Константные функции.
    \item Сложение.
    \item Умножение.
    \item Функцию $ifzero(n, m, p)$: если $n = 0$, то $m$, в противном случае $p$.
\end{enumerate}

Однако, типизуемые термы имеют не только вычислительную интерпретацию,
но и устанавливают соответствие между $\lambda$-исчислением и логикой.

\begin{theorem}[Соответствие Карри \dash Ховарда \cite{SorUrz06}]
    \label{CarryHowardTheorem}
    В исчислении $\lambda_\rightarrow$ суждение $\Gamma \vdash N : \phi$ выводится тогда и только
    тогда, когда $\operatorname{range}(\Gamma) \vdash \phi$ верно для импликативного фрагмента логики высказываний $IPC$,
    где $\operatorname{range}$ \dash функция из множества пар переменных и типов в множество формул логики высказываний:
    $$\operatorname{range} : (x : \psi) \mapsto \psi$$
\end{theorem}

Соответствие Карри \dash Ховарда дает старт парадигме "утверждения-как-типы": мы берем некоторое типизованное
$\lambda$-исчисление, интерпретируем типы как формальные утверждения (теоремы), а соответствующие
$\lambda$-термы \dash как доказательства. Причем такой подход заведома конструктивен: $\lambda$-исчисление
продолжает сохранять некоторую вычислительную интерпретацию, а значит термы-доказательства
являются конструкциями в стиле интерпретации Брауера \dash Гейтинга \dash Колмогорова.

Усложняя язык $\lambda$-термов и типов и расширяя правила типизации можно получать все более выразительные
исчисления, которые, с одной стороны, устанавливают соответствия со все более содержательными логиками,
а с другой стороны, типизуют все большие подмножества тотальных вычислимых функций,
сохраняя все хорошие мета-свойства, такие как сильная нормализуемость и конфлюэнтность. Двигаясь по первому пути,
можно пополнить язык термов конструкциями прямой суммы и прямого произведения, а также константой $\bot$,
установив соответствие между $\lambda$-исчислением и интуиционистской логикой высказываний $Int$.
А разрешив определять типы через термы, и введя тем самым так называемые зависимые типы, можно
получить исчисление $\lambda P_1$, которому соответствует интуиционистская логика предикатов.

Двигаясь же по второму пути, мы можем разрешить навешивать кванторы на типы, получив так называемую Систему $F$.
Чуть более формально, язык системы $F$ позволяет писать термы вида 
$\Lambda p . M$
и
$M \phi$
где $\alpha$ \dash это тип. Иными словами, в системе $F$ разрешены не только $\lambda$-абстракции
и $\lambda$-применения по термам, но и по типам. Соответствующая грамматика термов принимает вид
$$t := x | (\lambda x : \phi . t) | (tt) | (\Lambda p . t) | (t \phi)$$
($x$ \dash $\lambda$-переменная, $p$ \dash типовая переменная, а $\phi$ \dash тип). Система переписывания термов
пополняется отношением $\beta'$-редукции
$$(\Lambda \alpha . M) \phi \rightarrow_{\beta'} M[\phi/\alpha]$$
Пример:
$$(\Lambda \alpha . \lambda f : \alpha \rightarrow \alpha . \lambda x : \alpha . f x) p \rightarrow_{\beta'} \lambda f : p \rightarrow p . \lambda x : p . f x$$
В типах появляется квантор всеобщности по типовым переменным
$$type := p | (type \rightarrow type) | (\forall p . type)$$
благодаря чему запись $\forall . p (p \rightarrow p)$, например, становится валидным типом.
Дедуктивная система вывода суждений о типизации Системы $F$ имеет следующий набор правил:
\begin{multicols}{2}
    \begin{prooftree}
        \AxiomC{}
            \RightLabel{$Var$}
        \UnaryInfC{$\Gamma, x : \phi \vdash x : \phi$}
    \end{prooftree}
    \begin{prooftree}
        \AxiomC{$\Gamma, x : \phi \vdash M : \psi$}
            \RightLabel{$Abs$}
        \UnaryInfC{$\Gamma \vdash (\lambda x : \phi . M) : \phi \rightarrow \psi$}
    \end{prooftree}
\end{multicols}
\begin{multicols}{2}
    \begin{prooftree}
        \AxiomC{$\Gamma \vdash M : \phi \rightarrow \psi$}
        \AxiomC{$\Gamma \vdash N : \phi$}
            \RightLabel{$App$}
        \BinaryInfC{$\Gamma \vdash (MN) : \psi$}
    \end{prooftree}
    \begin{prooftree}
        \AxiomC{$\Gamma \vdash M : \forall p.\phi$}
            \RightLabel{$Inst$}
        \UnaryInfC{$\Gamma \vdash (M \psi) : \phi[\psi/p]$}
    \end{prooftree}
\end{multicols}
\begin{prooftree}
    \AxiomC{$\Gamma \vdash M : \phi$}
        \RightLabel{(если $p \not \in FVar(\Gamma)$) $Gen$}
    \UnaryInfC{$\Gamma \vdash (\Lambda p . M) : \forall p . \phi$}
\end{prooftree}

Типизуемые термы Системы $F$ продолжают быть сильно-нормализуемыми\cite{SorUrz06}, однако им 
соответствует гораздо большее подмножество вычислимых функций, чем для типизуемых термов $\lambda_\rightarrow$.

\begin{theorem}[\cite{Gira71}]
    Пусть функция $g : \mathbb N \rightarrow \mathbb N$ определима в арифметике Пеано второго порядка.
    Тогда в Системе $F$ существует такой типизуемый терм $G$, что
    $$G[n]_F \rightarrow_{\alpha \beta \beta'} [g(n)]_F$$
    для любого $n \in \mathbb N$ (здесь $[n]_F$ \dash нумерал для числа $n$ в Системе $F$).
\end{theorem}

\subsection{Теория типов Мартин-Лёфа}

Интуиционистская теория типов Мартин-Лёфа $MLTT$ является типизованным $\lambda$-исчислением, в котором,
с одной стороны, типизуем довольно большой класс вычислимых функций, а с другой стороны, соответствующий
язык термов типов достаточно богат, чтобы претендовать на формальные основания конструктивной
математики в парадигме "утверждения как типы". $MLTT$ не соответствует напрямую никакой интуиционистской логике
(в смысле соответствия Карри \dash Ховарда), являясь самостоятельной формальной системой.

Язык $\lambda$-термов $MLTT$, по мимо стандартных конструкций простого $\lambda$-исчисления, содержит
примитивные константы $c, c', \dots$ и функциональные константы $f, f', \dots$, которые мы будем вводить
далее по ходу повествования. Грамматика термов $MLTT$ имеет вид
$$t := x | \lambda x : A . t | t(t') | c | f$$
Здесь $t(t')$ \dash запись $\lambda$-аппликации в более привычном математическом виде применения функции к аргументу,
такая нотация является более подходящей для $MLTT$ из-за усложнения синтаксиса.

Язык термов не определяется отдельно, вместо этого мы считаем, что язык типов совпадает с языком термов.
Это означает, что у нас появляются специальные термы, которые, в каком-то смысле, являются именами типов.
Такие термы-имена мы будем обозначать теперь заглавными латинскими буквами $A, A', B, \dots$.
Первым примером такого рода термов является счетное семейство примитивных констант
$$\mathcal U_0, \mathcal U_1, \dots, \mathcal U_i, \dots$$
Константа $\mathcal U_i$ означает универсум типов уровня $i$, а запись $A : \mathcal U_i$ означает,
что терм $A$ является именем типа (или просто типом) из $i$-ого универсума.

В отличие от $\lambda_\rightarrow$, мы не будем раздельно определять понятие контекста, структуры формальной
системы переписывания термов и исчисление суждений о типизации, а объединим всё это
в одну большую дедуктивную систему, похожую на исчисление естественного вывода. В этой системе есть
три вида суждений:
\begin{enumerate}
    \item $\Gamma \; ctx$ \dash контекст $\Gamma$ корректно определён.
    \item $\Gamma \vdash a : A$ \dash в контексте $\Gamma$ терм $a$ имеет тип $A$.
    \item $\Gamma \vdash a \doteq b : A$ \dash в контексте $\Gamma$ $a$ и $b$ являются
    эквивалентными термами типа $A$.
\end{enumerate}
Далее мы перечислим правила вывода, позволяющие получать подобные суждения.

Во-первых, мы хотим считать пустой контекст корректным, и выражаем это через правило
\begin{prooftree}
    \AxiomC{}
        \RightLabel{$ctx-EMP$}
    \UnaryInfC{$\cdot  \; ctx$}
\end{prooftree}
Далее мы описываем способ получения корректного контекста пополнением парами $x : A$
\begin{prooftree}
    \AxiomC{$x_1 : A_1, \dots, x_n : A_n \vdash A_{n+1} : \mathcal U_i$}
        \RightLabel{$ctx-EXT$}
    \UnaryInfC{$(x_1 : A_1, \dots, x_n : A_n, x_{n+1} : A_{n+1}) \; ctx$}
\end{prooftree}
Заметим, что теперь контекст \dash это упорядоченная конечная последовательность пар: тип $A_n$ переменной $x_n$
сам является термом, и может зависеть от свободных переменных, но только тех, которое до этого
уже были добавлены в контекст. Более того, одна и та же переменная может встречаться
в последовательности несколько раз. Однако одной и той же переменной в контексте 
все еще нельзя присвоить два различных типа (не равных как слова нашего языка).

Введем аналог правила $Var$ исчисления $\lambda_\rightarrow$ для $MLTT$:
\begin{prooftree}
    \AxiomC{$(x_1 : A_n, \dots, x_i : A_i, \dots, x_n : A_n) \; ctx$}
    \RightLabel{$Var$}
    \UnaryInfC{$x_1 : A_n, \dots, x_i : A_i, \dots, x_n : A_n \vdash x_i : A_i$}
\end{prooftree}

Теперь введем правила, задающие полиморфизм универсумов типов $\mathcal U_i$.
\begin{multicols}{2}
    \begin{prooftree}
        \AxiomC{$\Gamma \; ctx$}
            \RightLabel{$\mathcal U-INTRO$}
        \UnaryInfC{$\Gamma \vdash \mathcal U_i :\mathcal U_{i+1}$}
    \end{prooftree}
    \begin{prooftree}
        \AxiomC{$\Gamma \vdash A : \mathcal U_i$}
            \RightLabel{$\mathcal U-CUMUL$}
        \UnaryInfC{$\Gamma \vdash A : \mathcal U_{i+1}$}
    \end{prooftree}
\end{multicols}
Правило $\mathcal U-INTRO$ в частности гласит, что каждая примитивная константа $\mathcal U_i$ сама является
типом в смысле замечания из начала данного раздела.

Далее, зададим базовые свойства отношения эквивалентности термов.
\begin{multicols}{2}
    \begin{prooftree}
        \AxiomC{$\Gamma \vdash a : A$}
        \UnaryInfC{$\Gamma \vdash a \doteq a : A$}
    \end{prooftree}
    \begin{prooftree}
        \AxiomC{$\Gamma \vdash a \doteq b : A$}
        \UnaryInfC{$\Gamma \vdash b \doteq a : A$}
    \end{prooftree}
\end{multicols}
\begin{prooftree}
        \AxiomC{$\Gamma \vdash a \doteq b : A$}
        \AxiomC{$\Gamma \vdash b \doteq c : A$}
        \BinaryInfC{$\Gamma \vdash a \doteq c : A$}
\end{prooftree}
\begin{multicols}{2}
    \begin{prooftree}
        \AxiomC{$\Gamma \vdash a : A$}
        \AxiomC{$\Gamma \vdash A \doteq B : \mathcal U_i$}
        \BinaryInfC{$\Gamma \vdash a : B$}
    \end{prooftree}
    \begin{prooftree}
        \AxiomC{$\Gamma \vdash a \doteq b : A$}
        \AxiomC{$\Gamma \vdash A \doteq B : \mathcal U_i$}
        \BinaryInfC{$\Gamma \vdash a \doteq b : B$}
    \end{prooftree}
\end{multicols}

Следующий набор правил принято называть структурой теории типов, так как они задают свойства контекстов
и подстановок. Во-первых, имеем два правила ослабления, которые говорят, что выводимость суждений
сохраняется при расширениях контекстов:
\begin{multicols}{2}
    \begin{prooftree}
        \AxiomC{$\Gamma \vdash A : \mathcal U_i$}
        \AxiomC{$\Gamma, \Delta \vdash b : B$}
            \RightLabel{$Wkg_1$}
        \BinaryInfC{$\Gamma, x : A, \Delta \vdash b : B$}
    \end{prooftree}
    \begin{prooftree}
        \AxiomC{$\Gamma \vdash A : \mathcal U_i$}
        \AxiomC{$\Gamma, \Delta \vdash a \doteq b : B$}
            \RightLabel{$Wkg_2$}
        \BinaryInfC{$\Gamma, x : A, \Delta \vdash a \doteq b : B$}
    \end{prooftree}
\end{multicols}
Во-вторых, имеем правила подстановок:
\begin{multicols}{2}
    \begin{prooftree}
        \AxiomC{$\Gamma \vdash a : A$}
        \AxiomC{$\Gamma, x : A, \Delta \vdash b : B$}
            \RightLabel{$Subst_1$}
        \BinaryInfC{$\Gamma, \Delta[a/x] \vdash b[a/x] : B[a/x]$}
    \end{prooftree}
    \begin{prooftree}
        \AxiomC{$\Gamma \vdash a : A$}
        \AxiomC{$\Gamma, x : A, \Delta \vdash b \doteq c : B$}
            \RightLabel{$Subst_2$}
        \BinaryInfC{$\Gamma, \Delta[a/x] \vdash b[a/x] \doteq c[a/x] : B[a/x]$}
    \end{prooftree}
\end{multicols}
\begin{prooftree}
    \AxiomC{$\Gamma \vdash a \doteq b : A$}
    \AxiomC{$\Gamma, x : A, \Delta \vdash c : C$}
        \RightLabel{$Subst_3$}
    \BinaryInfC{$\Gamma, \Delta[a/x] \vdash c[a/x] \doteq c[b/x] : C[a/x]$}
\end{prooftree}
Естественно, во всех правилах $Subst_i$ подстановка должна быть корректной: никакая переменная, которая до
подстановки была свободной, не должна стать связанной. 

Теперь мы можем перейти к обсуждению содержательных типов. $\Pi$-тип, также известный как тип зависимого
произведения, является обобщением стрелочного типа $\phi \rightarrow \psi$ из системы $\lambda_\rightarrow$.
По правилу $Abs$, если в предположении, что переменная $x$ имеет тип $A$, мы можем заключить, что терм $t$
имеет тип $B$, то мы можем с помощью $\lambda$-абстракции получить тип $A \rightarrow B$. Однако в $MLTT$
тип $B$ сам является термом, а значит сам может зависеть от свободной переменной $x$, мы должны учесть это
в наших правилах вывода. Для этого мы вводим функциональную константу $\Pi$ и правило формирование типа
зависимого произведения:
\begin{prooftree}
    \AxiomC{$\Gamma \vdash A : \mathcal U_i$}
    \AxiomC{$\Gamma, x : A \vdash b : B$}
        \RightLabel{$\Pi-FORM$}
    \BinaryInfC{$\Gamma \vdash (\prod \limits_{x:A} B) : \mathcal U_i$}
\end{prooftree}

Затем мы вводим правило введение $\Pi$-типа, которое описывает способ получение термов типа зависимого
произведение (аналог правила $Abs$):
\begin{prooftree}
    \AxiomC{$\Gamma, x : A \vdash b : B$}
        \RightLabel{$\Pi-INTRO$}
    \UnaryInfC{$\Gamma \vdash (\lambda x : A . b) : \prod \limits_{x : A} B$}
\end{prooftree}
Причем, как и в случае $\lambda_\rightarrow$, $\lambda$-абстракция связывает свободные вхождения переменной
$x$ в терм  $b$.

Далее, введем правило элиминации, описывающие способ получения новых типов, если у нас уже есть терма
типа зависимого произведения.
\begin{prooftree}
    \AxiomC{$\Gamma \vdash f : \prod \limits_{x : A} B$}
    \AxiomC{$\Gamma \vdash a : A$}
        \RightLabel{$\Pi-ELIM$}
    \BinaryInfC{$\Gamma \vdash f(a) : B[a/x]$}
\end{prooftree}

Наконец, введем вычислительное правило, описывающее класс эквивалентности термов $\Pi$-типа (аналог 
рефлексивно-транзитивного замыкания отношения $\beta$-редукции):
\begin{prooftree}
    \AxiomC{$\Gamma, x : A \vdash b : B$}
    \AxiomC{$\Gamma \vdash a : A$}
        \RightLabel{$\Pi-COMP$}
    \BinaryInfC{$\Gamma \vdash (\lambda x : A . b)(a) \doteq b[a/x] : B[a/x]$}
\end{prooftree}
Причем подстановка в правиле $\Pi-COMP$, как и раньше, должна быть корректной.

Следующий набор правил устанавливает взаимодействие $\Pi$-типов с отношением $\doteq$:
\begin{prooftree}
    \AxiomC{$\Gamma \vdash A : \mathcal U_I$}
    \AxiomC{$\Gamma, x : A \vdash B \doteq B' : \mathcal U_i$}
        \RightLabel{$\Pi-FORM-EQ$}
    \BinaryInfC{$\Gamma \vdash (\prod \limits_{x : A} B) \doteq (\prod \limits_{x : A} B') : \mathcal U_i$}
\end{prooftree}
\begin{prooftree}
    \AxiomC{$\Gamma, x : A \vdash b \doteq b' : B$}
        \RightLabel{$\Pi-INTRO-EQ$}
    \UnaryInfC{$\Gamma \vdash (\lambda x : A . b) \doteq (\lambda x : A . b') : \prod \limits_{x : A} B$}
\end{prooftree}
\begin{prooftree}
    \AxiomC{$\Gamma \vdash f : \prod \limits_{x : A} B$}
    \AxiomC{$\Gamma \vdash a \doteq a' : A$}
        \RightLabel{$\Pi-ELIM-EQ$}
    \BinaryInfC{$\Gamma \vdash f(a) \doteq f(a') : B[a/x]$}
\end{prooftree}

Приведем пример вывода какого-либо суждения о типе зависимого произведения. Покажем, например, что
правило
\begin{prooftree}
\AxiomC{$\Gamma \vdash A : \mathcal U_i$}
\AxiomC{$\Gamma, x : A \vdash b : B$}
\BinaryInfC{$\Gamma \vdash (\lambda x : A . (\lambda x : A . b)(x)) \doteq (\lambda x : A . b) : \prod \limits_{x : A} B$}
\end{prooftree} 
является допустимым (получается как композиция других правил). Действительно,
\begin{prooftree}
    \AxiomC{$\Gamma \vdash A : \mathcal U_i$}
        \RightLabel{$ctx-EXT$}
    \UnaryInfC{($\Gamma, x : A) \; ctx$}
        \RightLabel{$Var$}
    \UnaryInfC{$\Gamma, x : A \vdash x : A$}
    \AxiomC{$\Gamma, x : A, \vdash b : B$}
        \RightLabel{$Wkg_1$}
    \UnaryInfC{$\Gamma, x : A, x : A \vdash b : B$}
        \RightLabel{$\Pi-COMP$}
    \BinaryInfC{$\Gamma, x : A \vdash (\lambda x : A . b)(x) \doteq b : B$}
        \RightLabel{$\Pi-INTRO-EQ$}
    \UnaryInfC{$\Gamma \vdash (\lambda x : A . (\lambda x : A . b)(x)) \doteq (\lambda x : A . b) : \prod \limits_{x : A} B$}
\end{prooftree}

С точки зрения формализации математики, типам зависимого произведения соответствуют одновременно квантор
всеобщности и импликация в смысле интерпретации Брауэра \dash Гейтинга \dash Колмогорова: если
зависимый тип $B(a)$ выражает некоторое свойство терма $a$, то тип $\prod_{(x : A)} B$ выражает
способ получить терм-доказательство свойства $B(x)$ для любого $x$ из типа $A$.

Введем теперь семейство индуктивных типов. Каждый индуктивный тип определяется правилами
формации, введения, элиминации и вычисления, а также правилами $FORM-EQ$, $INTRO-EQ$ и $ELIM-EQ$,
по аналогии с правилами $\Pi-FORM-EQ$, $\Pi-INTRO-EQ$ и $\Pi-ELIM-EQ$ соответственно
(мы не будем выписывать такие правило явно из соображения экономии, в силу их простого устройства).
Начнем с типа $\mathds 1$, населенного единственной примитивной константой $*$.
\begin{multicols}{2}
    \begin{prooftree}
        \AxiomC{$\Gamma \; ctx$}
            \RightLabel{$\mathds 1-INTRO$}
        \UnaryInfC{$\Gamma \vdash \mathds 1 : \mathcal U_0$}
    \end{prooftree}
    \begin{prooftree}
        \AxiomC{$\Gamma \; ctx$}
            \RightLabel{$\mathds 1-INTRO$}
        \UnaryInfC{$\Gamma \vdash * : \mathds 1$}
    \end{prooftree}
\end{multicols}
    \begin{prooftree}
        \AxiomC{$\Gamma, x : \mathds 1 \vdash C:\mathcal U_i$}
        \AxiomC{$\Gamma \vdash c : C[*/x]$}
        \AxiomC{$\Gamma \vdash a : \mathds 1$}
            \RightLabel{$\mathds 1-ELIM$}
        \TrinaryInfC{$\Gamma \vdash \ind_{\mathds 1}(x.C, c, a) : C[a/x]$}
    \end{prooftree}    
    \begin{prooftree}
        \AxiomC{$\Gamma, x : \mathds 1 \vdash C : \mathcal U_i$}
        \AxiomC{$\Gamma \vdash c : C[*/x]$}
            \RightLabel{$\mathds 1-COMP$}
        \BinaryInfC{$\Gamma \vdash \ind_{\mathds 1}(x.C, c, *) \doteq c : C[*/x]$}
    \end{prooftree}

Правила $\mathds 1-ELIM$ и $\mathds 1-COMP$ вводят функциональную константу $\ind_{\mathds 1}$, называемую
индуктором, которая, с одной стороны, позволяет получить тип $C(x)$ для любого терма $x$ типа $\mathds 1$,
в предположение, что мы можем построить терм $c$ конкретного типа $C[*/x]$, а с другой стороны, позволяет
задавать тотальные рекурсивные функции от термов типа $\mathds 1$. Запись $x.C$ является сокращением
для терма $(\lambda x : \mathds 1 . C) : \prod_{(x : \mathds 1)} \mathcal U_i$, и подчеркивает, что
все вхождения $x$ в $C$ являются связанными.

Тип $\mathds 1$ является вырожденным примером индуктивного типа, так мы постулируем в нем наличие всего одного
элемента, и на его примере трудно увидеть насколько мощными конструкциями являются индуктивные типы. 
Более иллюстративным примером является тип $\mathbb B$, который населен двумя примитивными константами
$\mathbb T$ и $\mathbb F$.
\begin{multicols}{3}
    \begin{prooftree}
        \AxiomC{$\Gamma \; ctx$}
            \RightLabel{$\mathbb B-FORM$}
        \UnaryInfC{$\Gamma \vdash \mathbb B : \mathcal U_0$}
    \end{prooftree}
    \begin{prooftree}
        \AxiomC{$\Gamma \; ctx$}
            \RightLabel{$\mathbb B - INTRO_1$}
        \UnaryInfC{$\Gamma \vdash \mathbb F : \mathbb B$}
    \end{prooftree}
    \begin{prooftree}
        \AxiomC{$\Gamma \; ctx$}
            \RightLabel{$\mathbb B - INTRO_2$}
        \UnaryInfC{$\Gamma \vdash \mathbb T : \mathbb B$}
    \end{prooftree}
\end{multicols}
\begin{prooftree}
    \AxiomC{$\Gamma, x : \mathbb B \vdash C : \mathcal U_i$}
    \AxiomC{$\Gamma \vdash a : C[\mathbb F/x]$}
    \AxiomC{$\Gamma \vdash b : C[\mathbb T/x]$}
    \AxiomC{$\Gamma \vdash c : \mathbb B$}
        \RightLabel{$\mathbb B - ELIM$}
    \QuaternaryInfC{$\Gamma \vdash \ind_{\mathbb B}(x.C, a, b, c) : C[c/x]$}
\end{prooftree}
\begin{prooftree}
    \AxiomC{$\Gamma, x : \mathbb B \vdash C : \mathcal U_i$}
    \AxiomC{$\Gamma \vdash a : C[\mathbb F/x]$}
    \AxiomC{$\Gamma \vdash b : C[\mathbb T/x]$}
        \RightLabel{$\mathbb B - COMP_1$}
    \TrinaryInfC{$\Gamma \vdash \ind_{\mathbb B}(x.C, a, b, \mathbb F)  \doteq a : C[\mathbb F/x]$}
\end{prooftree}
\begin{prooftree}
    \AxiomC{$\Gamma, x : \mathbb B \vdash C : \mathcal U_i$}
    \AxiomC{$\Gamma \vdash a : C[\mathbb F/x]$}
    \AxiomC{$\Gamma \vdash b : C[\mathbb T/x]$}
        \RightLabel{$\mathbb B - COMP_2$}
    \TrinaryInfC{$\Gamma \vdash \ind_{\mathbb B}(x.C, a, b, \mathbb T)  \doteq b : C[\mathbb T/x]$}
\end{prooftree}

Рассмотрим $\mathbb B$ как булеву алгебру из двух элементов. Покажем, что терм
$$\lambda x : \mathbb B . \ind_{\mathbb B}(y.B, \mathbb T, \mathbb F, x)$$
задает функций отрицания на этой алгебре: применение этого терма к $\mathbb F$ эквивалентно константе $\mathbb T$,
а применение к $\mathbb T$ \dash константе $\mathbb F$. Для начала покажем, что наша
$\lambda$-абстракция имеет ожидаемый тип $\prod_{(x : \mathbb B)} \mathbb B$
(переводит термы типа $\mathbb B$ в термы типа $\mathbb B$):
\begin{prooftree}
    \AxiomC{$\dots$}
    \UnaryInfC{$x : \mathbb B, y : \mathbb B \vdash \mathbb B : \mathcal U_0$}
    \AxiomC{$\dots$}
    \UnaryInfC{$x : \mathbb B \vdash \mathbb T : \mathbb B$}
    \AxiomC{$\dots$}
    \UnaryInfC{$x : \mathbb B \vdash \mathbb F : \mathbb B$}
    \AxiomC{$\dots$}
    \UnaryInfC{$x : \mathbb B \vdash x : \mathbb B$}
        \RightLabel{$\mathbb B - ELIM$}
    \QuaternaryInfC{$x : \mathbb B \vdash \ind_{\mathbb B}(y.\mathbb B, \mathbb T, \mathbb F, x) : \mathbb B$}
    \UnaryInfC{$\vdash (\lambda x : \mathbb B . \ind_{\mathbb B}(y.\mathbb B, \mathbb T, \mathbb F, x)) : \prod \limits_{x : \mathbb B} \mathbb B$}
\end{prooftree}
Теперь покажем, что такая $\lambda$-абстракция действительно вычисляет функцию отрицания.
Действительно, с одной стороны
\begin{prooftree}
    \AxiomC{$\dots$}
    \UnaryInfC{$y : \mathbb B \vdash \mathbb B : \mathcal U_0$}
    \AxiomC{$\dots$}
    \UnaryInfC{$\vdash \mathbb T : \mathbb B$}
    \AxiomC{$\dots$}
    \UnaryInfC{$\vdash \mathbb F : \mathbb B$}
        \RightLabel{$\mathbb B - COMP_1$}
    \TrinaryInfC{$\vdash \ind_{\mathbb B}(y.\mathbb B, \mathbb T, \mathbb F, \mathbb F) \doteq \mathbb T : \mathbb B$}
\end{prooftree}
и
\begin{prooftree}
    \AxiomC{$\dots$}
    \UnaryInfC{$y : \mathbb B \vdash \mathbb B : \mathcal U_0$}
    \AxiomC{$\dots$}
    \UnaryInfC{$\vdash \mathbb T : \mathbb B$}
    \AxiomC{$\dots$}
    \UnaryInfC{$\vdash \mathbb F : \mathbb B$}
        \RightLabel{$\mathbb B - COMP_2$}
    \TrinaryInfC{$\vdash \ind_{\mathbb B}(y.\mathbb B, \mathbb T, \mathbb F, \mathbb T) \doteq \mathbb F : \mathbb B$}
\end{prooftree}
С другой стороны, выводим
\begin{prooftree}
    \AxiomC{$\dots$}
    \UnaryInfC{$x : \mathbb B \vdash \ind_{\mathbb B}(y.\mathbb B, \mathbb T, \mathbb F, x) : \mathbb B$}
    \AxiomC{$\dots$}
    \UnaryInfC{$\vdash \mathbb F : \mathbb B$}
        \RightLabel{$\Pi-COMP$}
    \BinaryInfC{$\vdash (\lambda x : \mathbb B . \ind_{\mathbb B}(y.\mathbb B, \mathbb T, \mathbb F, x))(\mathbb F) \doteq \ind_{\mathbb B}(y.\mathbb B, \mathbb T, \mathbb F, \mathbb F) : \mathbb B$}
\end{prooftree}
и
\begin{prooftree}
    \AxiomC{$\dots$}
    \UnaryInfC{$x : \mathbb B \vdash \ind_{\mathbb B}(y.\mathbb B, \mathbb T, \mathbb F, x) : \mathbb B$}
    \AxiomC{$\dots$}
    \UnaryInfC{$\vdash \mathbb T : \mathbb B$}
        \RightLabel{$\Pi-COMP$}
    \BinaryInfC{$\vdash (\lambda x : \mathbb B . \ind_{\mathbb B}(y.\mathbb B, \mathbb T, \mathbb F, x))(\mathbb T) \doteq \ind_{\mathbb B}(y.\mathbb B, \mathbb T, \mathbb F, \mathbb T) : \mathbb B$}
\end{prooftree}
А значит, пользуясь базовым правилом транзитивности для $\doteq$, мы можем вывести суждения
$$\vdash (\lambda x : \mathbb B . \ind_{\mathbb B}(y.\mathbb B, \mathbb T, \mathbb F, x))(\mathbb F) \doteq \mathbb T : \mathbb B$$ 
и
$$\vdash (\lambda x : \mathbb B . \ind_{\mathbb B}(y.\mathbb B, \mathbb T, \mathbb F, x))(\mathbb T) \doteq \mathbb F : \mathbb B$$ 

Наиболее хорошо, пожалуй, всю вычислительную мощь индукторов можно показать на типе натуральных чисел $\mathbb N$.
\begin{multicols}{2}
    \begin{prooftree}
        \AxiomC{$\Gamma \; ctx$}
            \RightLabel{$\mathbb N - FORM$}
        \UnaryInfC{$\Gamma \vdash \mathbb N : \mathcal U_0$}
    \end{prooftree}
    \begin{prooftree}
        \AxiomC{$\Gamma \; ctx$}
            \RightLabel{$\mathbb N - INTRO_1$}
        \UnaryInfC{$\Gamma \vdash zero : \mathbb N$}
    \end{prooftree}
\end{multicols}
\begin{prooftree}
    \AxiomC{$\Gamma \vdash n : \mathbb N$}
        \RightLabel{$\mathbb N - INTRO_2$}
    \UnaryInfC{$\Gamma \vdash suc(n) : \mathbb N$}
\end{prooftree}
\begin{prooftree}
    \AxiomC{$\Gamma, x : \mathbb N \vdash C : \mathcal U_i$}
    \AxiomC{$\Gamma \vdash a : C[0/x]$}
    \AxiomC{$\Gamma, x : \mathbb N, y : C \vdash b : C[suc(x)/x]$}
    \AxiomC{$\Gamma \vdash n : \mathbb N$}
    \QuaternaryInfC{$\Gamma \vdash \ind_{\mathbb N}(x.C, a, x.y.b, n) : C[n/x]$}
\end{prooftree}
$$(\mathbb N - ELIM)$$
\begin{prooftree}
    \AxiomC{$\Gamma, x : \mathbb N \vdash C : \mathcal U_i$}
    \AxiomC{$\Gamma \vdash a : C[zero/x]$}
    \AxiomC{$\Gamma, x : \mathbb N, y : C \vdash b : C[suc(x)/x]$}
        \RightLabel{$\mathbb N - COMP_1$}
    \TrinaryInfC{$\Gamma \vdash \ind_{\mathbb N}(x.C, a, x.y.b, zero) \doteq a : C[zero/x]$}
\end{prooftree}
\begin{prooftree}
    \AxiomC{$\Gamma, x : \mathbb N \vdash C : \mathcal U_i$}
    \AxiomC{$\Gamma \vdash a : C[zero/x]$}
    \AxiomC{$\Gamma, x : \mathbb N, y : C \vdash b : C[suc(x)/x]$}
    \AxiomC{$\Gamma \vdash n : \mathbb N$}
    \QuaternaryInfC{$\Gamma \vdash \ind_{\mathbb N}(x.C, a, x.y.b, suc(n)) \doteq b[n/x][(\ind_{\mathbb N}(x.C, a, x.y.b, n))/y] : C[suc(n)/x]$}
\end{prooftree}
$$(\mathbb N - COMP_2)$$

Тип $\mathbb N$ населён примитивной константой $zero$, а также снабжён функциональной константой $suc$, которая
позволяет получать последователей $n+1$. Правило $\mathbb N - ELIM$ гласит, что для любого
типа $C$, если можно предъявить терм типа $C(zero)$ и, в предположении что $x$ имеет тип $\mathbb N$, а
$y$ имеет тип $C(x)$, то с помощью индуктора $\ind_{\mathbb N}$ можно получить терм типа $C(n)$ для любого
$n$ из типа $\mathbb N$. Правила $\mathbb N - COMP_1$ и $\mathbb N - COMP_2$ же гласят, что с вычислительной
точки зрения индукторы задают в точности тотальные рекурсивные функции на натуральных числах:
$$f(0) = c$$
$$f(n+1) = g(f(n))$$
В качестве игрушечного примера, покажем, что с помощью $\ind_{\mathbb N}$ можно построить функцию
$n \mapsto n + 1$ (с более содержательными примерами мы столкнемся, начав работу с Agda). Действительно,
\begin{prooftree}
    \AxiomC{$\dots$}
    \UnaryInfC{$x : \mathbb N \vdash \mathbb N : \mathcal U_0$}
    \AxiomC{$\dots$}
    \UnaryInfC{$\vdash suc(zero) : \mathbb N$}
    \AxiomC{$\dots$}
    \UnaryInfC{$x : \mathbb N, y : \mathbb N \vdash suc(y) : \mathbb N$}
    \TrinaryInfC{$\vdash \ind_{\mathbb N}(x.\mathbb N, suc(zero), x.y.suc(y), zero) \doteq suc(zero) : \mathbb N$}
\end{prooftree}
и
\begin{prooftree}
    \AxiomC{$\dots$}
    \UnaryInfC{$x : \mathbb N \vdash \mathbb N : \mathcal U_0$}
    \AxiomC{$\dots$}
    \UnaryInfC{$\vdash suc(zero) : \mathbb N$}
    \AxiomC{$\dots$}
    \UnaryInfC{$x : \mathbb N, y : \mathbb N \vdash suc(y) : \mathbb N$}
    \AxiomC{$\dots$}
    \UnaryInfC{$\vdash zero : \mathbb N$}
    \QuaternaryInfC{$\vdash \ind_{\mathbb N}(x.\mathbb N, suc(zero), x.y.suc(y), suc(zero)) \doteq suc(\ind_{\mathbb N}(x.\mathbb N, suc(zero), x.y.suc(y), zero)) : \mathbb N$}
\end{prooftree}
Объединим два этих вывода с помощью правила $\mathbb N-INTRO_2-EQ$ и выведем суждение
$$\vdash \ind_{\mathbb N}(x.\mathbb N, suc(zero), x.y.suc(y), suc(zero)) \doteq suc(suc(zero)) : \mathbb N$$
Далее, внешней индукцией по длине терма $n = \underbrace{suc(suc(\dots suc(}_{n \text{-раз}}zero)\dots)$
мы можем доказать выводимость суждений
$$\vdash \ind_{\mathbb N}(x.\mathbb N, suc(zero), x.y.suc(y), suc(suc(zero))) \doteq suc(suc(suc(zero))) : \mathbb N$$
$$\vdash \ind_{\mathbb N}(x.\mathbb N, suc(zero), x.y.suc(y), suc(suc(suc(zero)))) \doteq suc(suc(suc(suc(zero)))) : \mathbb N$$
$$\vdash \ind_{\mathbb N}(x.\mathbb N, suc(zero), x.y.suc(y), suc(suc(suc(suc(zero))))) \doteq suc(suc(suc(suc(suc(zero))))) : \mathbb N$$
$$\dots$$

Вернёмся к вопросу логических конструкций с точки зрения интерпретации Брауэра \dash Гейтинга \dash Колмогорова.
Согласно БГК, формуле $\exists x \phi(x)$, выражающей существование некоторого объекта $x$ для которого истинно
свойство $\phi(x)$. С теоретико-типовой точки зрения, для любого зависимого типа $P : \prod_{x : A} B$ 
нам  нужно уметь строить тип пар следующего вида: первым элементом пары является терм $a$ типа $A$, а вторым \dash
терм $b$ типа $B(a)$, который конструктивно доказывает, что для $a$ действительно верно свойство $P(a)$.
В $MLTT$ такой тип пар принято вводить функциональной константой $\Sigma$, называть типом зависимой пары
и описывать следующем набором правил:
\begin{prooftree}
    \AxiomC{$\Gamma \vdash A : \mathcal U_i$}
    \AxiomC{$\Gamma, x : A \vdash  B : \mathcal U_i$}
        \RightLabel{$\Sigma-FORM$}
    \BinaryInfC{$\Gamma \vdash (\sum \limits_{x : A} B) : \mathcal U_i$}
\end{prooftree}
\begin{prooftree}
    \AxiomC{$\Gamma, x : A \vdash B : \mathcal U_i$}
    \AxiomC{$\Gamma \vdash a : A$}
    \AxiomC{$\Gamma \vdash b : B[a/x]$}
        \RightLabel{$\Sigma-INTRO$}
    \TrinaryInfC{$\Gamma \vdash (a, b) : \sum \limits_{x : A} B$}
\end{prooftree}
\begin{prooftree}
    \AxiomC{$\Gamma, z : \sum \limits_{x : A} B \vdash C : \mathcal U_i$}
    \AxiomC{$\Gamma, x : A, y : B \vdash g : C[(x,y)/z]$}
    \AxiomC{$\Gamma \vdash p : \sum \limits_{x : A} B$}
        \RightLabel{$\Sigma-ELIM$}
    \TrinaryInfC{$\Gamma \vdash \ind_{\sum_{(x : A)} B}(z.C, x.y.g, p) : C[p/z]$}
\end{prooftree}
\begin{prooftree}
    \AxiomC{$\Gamma, z : \sum \limits_{x : A} B \vdash C : \mathcal U_i$}
    \AxiomC{$\Gamma, x : A, y : B \vdash g : C[(x,y)/z]$}
    \AxiomC{$\Gamma \vdash a : A$}
    \AxiomC{$\Gamma \vdash b : B[a/x]$}
        \RightLabel{$\Sigma-COMP$}
    \QuaternaryInfC{$\Gamma \vdash (\ind_{\sum_{(x : A)} B}(z.C, x.y.g, (a, b))) \doteq g[a/x][b/y] : C[(a, b)/z]$}
\end{prooftree}

Легко заметить, что если в зависимой сумме $\sum_{(x : A)} B$ тип $B$ не зависит от переменной $x$ как терм, то
$\sum_{(x : A)} B$ является теоретико-типовым аналогом прямого произведения. Естественно, мы также
можем определить тип ко-произведения $A \dotplus B$, который, в некотором смысле, населён дизъюнктивным
объединением термов типа $A$ и типа $B$:
\begin{prooftree}
    \AxiomC{$\Gamma \vdash A : \mathcal U_i$}
    \AxiomC{$\Gamma \vdash B : \mathcal U_i$}
        \RightLabel{$\dotplus-FORM$}
    \BinaryInfC{$\Gamma \vdash (A \dotplus B) : \mathcal U_i$}
\end{prooftree}
\begin{prooftree}
    \AxiomC{$\Gamma \vdash A : \mathcal U_i$}
    \AxiomC{$\Gamma \vdash B : \mathcal U_i$}
    \AxiomC{$\Gamma \vdash a : A$}
        \RightLabel{$\dotplus - INTRO_1$}
    \TrinaryInfC{$\Gamma \vdash inl(a) : A \dotplus B$}
\end{prooftree}
\begin{prooftree}
    \AxiomC{$\Gamma \vdash A : \mathcal U_i$}
    \AxiomC{$\Gamma \vdash B : \mathcal U_i$}
    \AxiomC{$\Gamma \vdash b : B$}
        \RightLabel{$\dotplus - INTRO_2$}
    \TrinaryInfC{$\Gamma \vdash inr(b) : A \dotplus B$}
\end{prooftree}
\begin{prooftree}
    \AxiomC{$\Gamma, x : A \vdash c : C[inl(x)/z]$}
    \alwaysNoLine
    \AxiomC{$\Gamma, z : (A \dotplus B) \vdash C : \mathcal U_i$}
    \UnaryInfC{$\Gamma, y : B \vdash d : C[inr(y)/z]$}
    \alwaysSingleLine
    \AxiomC{$\Gamma \vdash e : (A \dotplus B)$}
        \RightLabel{$\dotplus - ELIM$}
    \TrinaryInfC{$\Gamma \vdash \ind_{A \dotplus B}(z.C, x.c, y.d, e) : C[e/z]$}
\end{prooftree}
\begin{prooftree}
    \AxiomC{$\Gamma, x : A \vdash c : C[inl(x)/z]$}
    \alwaysNoLine
    \AxiomC{$\Gamma, z : (A \dotplus B) \vdash C : \mathcal U_i$}
    \UnaryInfC{$\Gamma, y : B \vdash d : C[inr(y)/z]$}
    \alwaysSingleLine
    \AxiomC{$\Gamma \vdash a : A$}
        \RightLabel{$\dotplus - COMP_1$}
    \TrinaryInfC{$\Gamma \vdash (\ind_{A \dotplus B}(z.C, x.c, y.d, inl(a))) \doteq c[a/x] : C[inl(a)/z]$}
\end{prooftree}
\begin{prooftree}
    \AxiomC{$\Gamma, x : A \vdash c : C[inl(x)/z]$}
    \alwaysNoLine
    \AxiomC{$\Gamma, z : (A \dotplus B) \vdash C : \mathcal U_i$}
    \UnaryInfC{$\Gamma, y : B \vdash d : C[inr(y)/z]$}
    \alwaysSingleLine
    \AxiomC{$\Gamma \vdash b : B$}
        \RightLabel{$\dotplus - COMP_2$}
    \TrinaryInfC{$\Gamma \vdash (\ind_{A \dotplus B}(z.C, x.c, y.d, inr(b))) \doteq d[b/y] : C[inr(b)/z]$}
\end{prooftree}

\AgdaTarget{\_≡\_}
\begin{code}

{-# OPTIONS --cubical #-}

open import Agda.Primitive using (Level; lzero; lsuc; _⊔_)
                           renaming (Set to 𝒰)

module MasterThesis where -- объявляем главный модуль

data _≡_ {l : Level} {A : 𝒰 l} : A → A → 𝒰 l where
    refl : (a : A) → a ≡ a

{-# BUILTIN EQUALITY _≡_  #-}

ap : ∀ {l m} → {A : 𝒰 l} → {B : 𝒰 m} → {x y : A} 
     → (f : A → B) → x ≡ y → f x ≡ f y
ap f (refl _) = refl _

\end{code}

Таким образом у нас есть индуктивный тип \AgdaDatatype{\_≡\_}

\begin{thebibliography}{}
    \bibitem[SorUrz06]{SorUrz06}
    M. H. Sorensen, P. Urzycyn. \textit{Lectures on the Curry-Howard \\ Isomorphism.}
    Studies in Logic and the Foundations of Mathematics. \\ Volume 149. 2006. ISBN
    978-0-444-52077-7.

    \bibitem[Gira71]{Gira71}
    J. Girard. \textit{Une Extension De ĽInterpretation De Gödel a ĽAnalyse, \\ Et Son Application  a ĽElimination Des Coupures Dans ĽAnalyse  \\ Et La Theorie Des Types.}
    Studies in Logic and the Foundations of Mathematics. Volume 63. 1971. ISBN 978-0-720-42259-7.

    \bibitem[Zakr07]{Zakr07}
    M. Zakrzewski. \textit{Definable functions in the simply typed lambda-calculus.} \\ 2007.
    arXiv:cs/0701022
\end{thebibliography}

\end{document}