\documentclass{article}[14pt]
\usepackage[utf8]{inputenc}
\usepackage[english,russian]{babel}
\usepackage{amsmath}
\usepackage{amsfonts}
\usepackage{hyphenat}
\usepackage{tikz-cd}
\usetikzlibrary{babel}
\usepackage{letltxmacro}
\usepackage{amsthm}
\usepackage{amssymb}
\usepackage{tikz-cd}
\usepackage{proof}
\usepackage{dirtytalk}
\usepackage[hidelinks]{hyperref}
\usepackage[links]{agda}
\usepackage{newunicodechar}
\usepackage{microtype}

\DisableLigatures[-]{encoding=T2A}

\newunicodechar{λ}{\ensuremath{\mathnormal\lambda}}
\newunicodechar{←}{\ensuremath{\mathnormal\from}}
\newunicodechar{→}{\ensuremath{\mathnormal\to}}
\newunicodechar{∀}{\ensuremath{\mathnormal\forall}}
\newunicodechar{𝒰}{\ensuremath{\mathcal{U}}}
\newunicodechar{≡}{\ensuremath{\mathnormal{\equiv}}}
\newunicodechar{⊔}{\ensuremath{\mathnormal{\sqcup}}}

\newtheorem{theorem}{Теорема}
\newtheorem{lemma}{Лемма}
\newtheorem{proposition}{Утверждение}
\newtheorem{definition}{Определение}
\newtheorem{corollary}{Следствие}

\title{Гомотопическая теория типов и ее модели}
\author{Глеб Красилич}
\date{Май 2023}

\begin{document}

\maketitle

\section{Теория типов Мартин-Лёфа}

Рассмотрим следующий код:

\AgdaTarget{\_≡\_}
\begin{code}

{-# OPTIONS --cubical #-}

open import Agda.Primitive using (Level; lzero; lsuc; _⊔_)
                           renaming (Set to 𝒰)

module MasterThesis where -- объявляем главный модуль

data _≡_ {l : Level} {A : 𝒰 l} : A → A → 𝒰 l where
    refl : (a : A) → a ≡ a

{-# BUILTIN EQUALITY _≡_  #-}

ap : ∀ {l m} → {A : 𝒰 l} → {B : 𝒰 m} → {x y : A} 
     → (f : A → B) → x ≡ y → f x ≡ f y
ap f (refl _) = refl _

\end{code}



Таким образом у нас есть индуктивный тип \AgdaDatatype{\_≡\_}

\end{document}