\documentclass{article}[14pt]
\usepackage[utf8]{inputenc}
\usepackage[english,russian]{babel}
\usepackage{amsmath}
\usepackage{amsfonts}
\usepackage{hyphenat}
\usepackage{tikz-cd}
\usetikzlibrary{babel}
\usepackage{letltxmacro}
\usepackage{amsthm}
\usepackage{amssymb}
\usepackage{tikz-cd}
\usepackage{proof}
\usepackage{dirtytalk}
\usepackage[hidelinks]{hyperref}
\usepackage[links]{agda}
\usepackage{newunicodechar}
\usepackage{microtype}

\DisableLigatures[-]{encoding=T2A}

\newunicodechar{λ}{\ensuremath{\mathnormal\lambda}}
\newunicodechar{←}{\ensuremath{\mathnormal\from}}
\newunicodechar{→}{\ensuremath{\mathnormal\to}}
\newunicodechar{∀}{\ensuremath{\mathnormal\forall}}
\newunicodechar{𝒰}{\ensuremath{\mathcal{U}}}
\newunicodechar{≡}{\ensuremath{\mathnormal{\equiv}}}
\newunicodechar{⊔}{\ensuremath{\mathnormal{\sqcup}}}

\newtheorem{theorem}{Теорема}
\newtheorem{lemma}{Лемма}
\newtheorem{proposition}{Утверждение}
\newtheorem{definition}{Определение}
\newtheorem{corollary}{Следствие}

\newcommand{\dash}{\textemdash\ }

\title{Гомотопическая теория типов и ее модели}
\author{Глеб Красилич}
\date{Май 2023}

\begin{document}

\maketitle

\section{Теории типов}

Гомотопическая теория типов (Homotopy Type Theory, HoTT) является расширением интуиционистской теории типов
Мартин-Лёфа (Martin-Löf Type Theory, MLTT), которая в свою очередь является выразительным
продолжением простого типизованного $\lambda$-исчисления до формальных оснований конструктивной математики.
Напомним основные факты о $\lambda$-исчислениях.

\subsection{Исчисление $\lambda_\rightarrow$}

Зафиксируем алфавит типовых переменных $TVar = \{p, q, v, \dots \}$ и алфавит $\lambda$-переменных
$\Lambda Var = \{x, y, z, \dots \}$
\begin{definition}
    Язык типов \dash это наименьшее множество $Types$ такое, что
    \begin{enumerate}
        \item $TVar \subset Types$.
        \item Если слова $\phi$ и $\psi$ принадлежат $Types$, то слово $(\phi \rightarrow \psi)$
        также принадлежит $Types$.
    \end{enumerate}
\end{definition}
Иными словами, язык типов можно задать грамматикой
$$type := pvar | (type \rightarrow type)$$
где $pvar \in TVar$. Если считать, что множество типовых переменных $TVar$ совпадает с множеством
пропозициональных переменных логики высказываний, то язык типов $Types$ в точности совпадает с
языком импликативного фрагмента (интуиционистской) логики высказываний $IPC$.  

\begin{definition}
    Язык термов просто типизованного по Чёрчу $\lambda$-исчисления \dash
    это наименьшее множество $\Lambda Terms$ такое, что
    \begin{enumerate}
        \item $\Lambda Var \subset \Lambda Term$.
        \item Если термы (слова) $M$ и $N$ принадлежат $\Lambda Terms$, то терм
        $(MN)$ также лежит в $\Lambda Terms$.
        \item Если $M \in \Lambda Terms$, то $(\lambda x : \phi . M) \in \Lambda Terms$
        для любой переменной $x \in \Lambda Var$ и любого типа $\phi \in Types$.
    \end{enumerate}
\end{definition}
Таким образом грамматика $\lambda$-термов имеет следующий вид:
$$t := x | (tt) | (\lambda x : \phi . t)$$
где $x$ \dash произвольная $\lambda$-переменная, а $\phi$ \dash произвольный тип.
Приведем несколько примеров грамматически корректных $\lambda$-термов:
$$x$$
$$((xx)x)$$
$$(\lambda x : p . x)$$
$$((\lambda x : (p \rightarrow q))y)$$

Естественно, скобки, необходимые для однозначности разбора, мы будем опускать для лучшей читаемости:
мы будем считать, что операция $\lambda$-абстракция (термы вида $\lambda x : \phi . M$) имеет более
низкий приоритет, чем $\lambda$-применение (термы вида $MN$), и не будем писать самые внешние скобки.
Мы также считаем, что при чтении терма, составленного с помощью операций равного приоритета,
скобки расставляются лево-ассоциативным образом, то есть, например, терм $xyz$ читается как $((xy)z)$
(в отличии от языка тиров/имликативного фрагмента логики высказываний, где скобки расставляются
право-ассоциативно: $p \rightarrow q \rightarrow v$ читается как $(p \rightarrow (q \rightarrow v))$).

Введем понятие свободных и связанных переменных.
\begin{definition}
    Определим рекурсивно функцию $FVar : \Lambda Terms \rightarrow \mathcal{P}(TVar)$,
    переводящую $\lambda$-терм в его множество так называемых свободных переменных:
    \begin{enumerate}
        \item $FVar(x) = \{ x \}$, если $x$ \dash $\lambda$-переменная.
        \item $FVar(MN) = FVar(M) \cup FVar(N)$.
        \item $FVar(\lambda x : \phi . M) = FVar(M) \setminus \{ x \}$.
    \end{enumerate}
    Если переменная $x$ входит в $\lambda$-терм $M$ как подтерм, но $x \not \in FVar(M)$, то $x$
    называется связанной переменной терма $M$. Терм беза свободных переменных называется
    комбинатором, или просто замкнутым термом.
\end{definition}

\AgdaTarget{\_≡\_}
\begin{code}

{-# OPTIONS --cubical #-}

open import Agda.Primitive using (Level; lzero; lsuc; _⊔_)
                           renaming (Set to 𝒰)

module MasterThesis where -- объявляем главный модуль

data _≡_ {l : Level} {A : 𝒰 l} : A → A → 𝒰 l where
    refl : (a : A) → a ≡ a

{-# BUILTIN EQUALITY _≡_  #-}

ap : ∀ {l m} → {A : 𝒰 l} → {B : 𝒰 m} → {x y : A} 
     → (f : A → B) → x ≡ y → f x ≡ f y
ap f (refl _) = refl _

\end{code}



Таким образом у нас есть индуктивный тип \AgdaDatatype{\_≡\_}

\end{document}