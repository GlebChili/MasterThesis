\documentclass{article}[12pt]
\usepackage[a4paper, left=30mm,
 top=30mm,  right=30mm,
 bottom=20mm]{geometry}
\usepackage[utf8]{inputenc}
\usepackage[english,russian]{babel}
\usepackage[hidelinks]{hyperref}
\usepackage{amsmath}
\usepackage{amsfonts}
\usepackage{hyphenat}
\usepackage{tikz-cd}
\usetikzlibrary{babel}
\usepackage{letltxmacro}
\usepackage{amsthm}
\usepackage{amssymb}
\usepackage{tikz-cd}
\usepackage{proof}
\usepackage{dirtytalk}
\usepackage[links]{agda}
\usepackage{newunicodechar}
\usepackage{microtype}
\usepackage{bussproofs}
\usepackage{multicol}
\usepackage{dsfont}
\usepackage{bbold}

\DisableLigatures[-]{encoding=T2A}

\newunicodechar{λ}{\ensuremath{\mathnormal\lambda}}
\newunicodechar{←}{\ensuremath{\mathnormal\from}}
\newunicodechar{→}{\ensuremath{\mathnormal\to}}
\newunicodechar{∀}{\ensuremath{\mathnormal\forall}}
\newunicodechar{𝒰}{\ensuremath{\mathcal{U}}}
\newunicodechar{≡}{\ensuremath{\mathnormal{\equiv}}}
\newunicodechar{⊔}{\ensuremath{\mathnormal{\sqcup}}}
\newunicodechar{₀}{\ensuremath{\mathnormal{{}_0}}}
\newunicodechar{₁}{\ensuremath{\mathnormal{{}_1}}}
\newunicodechar{₂}{\ensuremath{\mathnormal{{}_2}}}
\newunicodechar{₃}{\ensuremath{\mathnormal{{}_3}}}
\newunicodechar{₄}{\ensuremath{\mathnormal{{}_4}}}
\newunicodechar{₅}{\ensuremath{\mathnormal{{}_5}}}
\newunicodechar{₆}{\ensuremath{\mathnormal{{}_6}}}
\newunicodechar{₇}{\ensuremath{\mathnormal{{}_7}}}
\newunicodechar{₈}{\ensuremath{\mathnormal{{}_8}}}
\newunicodechar{₉}{\ensuremath{\mathnormal{{}_9}}}
\newunicodechar{𝟙}{\ensuremath{\mathbb{1}}}
\newunicodechar{𝔹}{\ensuremath{\mathbb{B}}}
\newunicodechar{𝔽}{\ensuremath{\mathbb{F}}}
\newunicodechar{𝕋}{\ensuremath{\mathbb{T}}}
\newunicodechar{Σ}{\ensuremath{\mathnormal{\Sigma}}}
\newunicodechar{ℕ}{\ensuremath{\mathbb{N}}}
\newunicodechar{∔}{\ensuremath{\mathnormal{\dotplus}}}
\newunicodechar{𝟘}{\ensuremath{\mathbb{0}}}
\newunicodechar{ᴺ}{\ensuremath{\mathnormal{{}^N}}}
\newunicodechar{·}{\ensuremath{\mathnormal{\cdot}}}
\newunicodechar{×}{\ensuremath{\mathnormal{\times}}}
\newunicodechar{¬}{\ensuremath{\mathnormal{\neg}}}
\newunicodechar{Π}{\ensuremath{\mathnormal{\Pi}}}

\newtheorem{theorem}{Теорема}
\newtheorem{lemma}{Лемма}
\newtheorem{proposition}{Утверждение}
\newtheorem{definition}{Определение}
\newtheorem{corollary}{Следствие}

\newcommand{\dash}{\textemdash\ }
\newcommand{\ind}{\operatorname{ind}}

\title{Гомотопическая теория типов и ее модели}
\author{Глеб Красилич}
\date{Май 2023}

\begin{document}

\maketitle

\section{Теории типов}

Гомотопическая теория типов (Homotopy Type Theory, HoTT) является расширением интуиционистской теории типов
Мартин-Лёфа (Martin-Löf Type Theory, MLTT), которая в свою очередь является выразительным
продолжением простого типизованного $\lambda$-исчисления до формальных оснований конструктивной математики.
Напомним основные факты о $\lambda$-исчислениях.

\subsection{Исчисление $\lambda_\rightarrow$}

Зафиксируем алфавит типовых переменных $TVar = \{p, q, v, \dots \}$ и алфавит $\lambda$-переменных
$\Lambda Var = \{x, y, z, \dots \}$
\begin{definition}
    Язык типов \dash это наименьшее множество $Types$ такое, что
    \begin{enumerate}
        \item $TVar \subset Types$.
        \item Если слова $\phi$ и $\psi$ принадлежат $Types$, то слово $(\phi \rightarrow \psi)$
        также принадлежит $Types$.
    \end{enumerate}
\end{definition}
Иными словами, язык типов можно задать грамматикой
$$type := pvar \; | \; (type \rightarrow type)$$
где $pvar \in TVar$. Если считать, что множество типовых переменных $TVar$ совпадает с множеством
пропозициональных переменных логики высказываний, то язык типов $Types$ в точности совпадает с
языком импликативного фрагмента (интуиционистской) логики высказываний $IPC$.  

\begin{definition}
    Язык термов просто типизованного по Чёрчу $\lambda$-исчисления \dash
    это наименьшее множество $\Lambda Terms$ такое, что
    \begin{enumerate}
        \item $\Lambda Var \subset \Lambda Term$.
        \item Если термы (слова) $M$ и $N$ принадлежат $\Lambda Terms$, то терм
        $(MN)$ также лежит в $\Lambda Terms$.
        \item Если $M \in \Lambda Terms$, то $(\lambda x : \phi . M) \in \Lambda Terms$
        для любой переменной $x \in \Lambda Var$ и любого типа $\phi \in Types$.
    \end{enumerate}
\end{definition}
Таким образом грамматика $\lambda$-термов имеет следующий вид:
$$t := x \; | \; (tt) \; | \; (\lambda x : \phi . t)$$
где $x$ \dash произвольная $\lambda$-переменная, а $\phi$ \dash произвольный тип.
Приведем несколько примеров грамматически корректных $\lambda$-термов:
$$x$$
$$((xx)x)$$
$$(\lambda x : p . x)$$
$$((\lambda x : (p \rightarrow q).y)z)$$

Естественно, скобки, необходимые для однозначности разбора, мы будем опускать для лучшей читаемости:
мы будем считать, что операция $\lambda$-абстракция (термы вида $\lambda x : \phi . M$) имеет более
низкий приоритет, чем $\lambda$-применение (термы вида $MN$), и не будем писать самые внешние скобки.
Мы также считаем, что при чтении терма, составленного с помощью операций равного приоритета,
скобки расставляются лево-ассоциативным образом, то есть, например, терм $xyz$ читается как $((xy)z)$
(в отличии от языка тиров/имликативного фрагмента логики высказываний, где скобки расставляются
право-ассоциативно: $p \rightarrow q \rightarrow v$ читается как $(p \rightarrow (q \rightarrow v))$).

Введем понятие свободных и связанных переменных.
\begin{definition}
    Определим рекурсивно функцию $FVar : \Lambda Terms \rightarrow \mathcal{P}(TVar)$,
    переводящую $\lambda$-терм в его множество так называемых свободных переменных:
    \begin{enumerate}
        \item $FVar(x) = \{ x \}$, если $x$ \dash $\lambda$-переменная.
        \item $FVar(MN) = FVar(M) \cup FVar(N)$.
        \item $FVar(\lambda x : \phi . M) = FVar(M) \setminus \{ x \}$.
    \end{enumerate}
    Если переменная $x$ входит в $\lambda$-терм $M$ как подтерм, но $x \not \in FVar(M)$, то $x$
    называется связанной переменной терма $M$. Терм беза свободных переменных называется
    комбинатором, или просто замкнутым термом.
\end{definition}

Теперь определим на $\lambda$-исчислении структуру формальной системы переписывания термов.
Напомним, что формальная система преобразований \dash это тройка $(A, \rightarrow, \twoheadrightarrow)$,
где $A$ \dash множество-носитель, $\rightarrow$ \dash бинарное отношение на $A$, а
$\twoheadrightarrow$ \dash транзитивно-рефлексивное замыкание $\rightarrow$. Если $A$ является
множеством слов некоторого формального языка, то тройку $(A, \rightarrow, \twoheadrightarrow)$ также
принято называть формальной системой переписывания термов, так как отношение $\rightarrow$
естественно воспринимать как правило преобразования слов.

Во-первых, нам бы хотелось иметь возможность заменять имена связанных переменных. Делается
это с помощью отношения $\alpha$-конверсии $\rightarrow_\alpha$: Если в терме $M$ есть подтерм вида
$\lambda x : \phi . N$, то $M \rightarrow_\alpha M'$, где терм $M'$ получается из $M$ заменой
какого либо единственного вхождения подтерма $\lambda x : \phi . N$ на подтерм $\lambda y : \phi . N[y/x]$, при условии, что
в $y \not \in FVar(N)$ (запись $N[y/x]$ означает, терм $N$ в котором все свободные вхождении переменной
$x$ заменены на переменную $y$). Приведем примеры термов, находящихся в отношении $\alpha$-конверсии:
$$(\lambda x : \phi . x) \rightarrow_\alpha (\lambda y : \phi . y)$$
$$((\lambda x : \phi . y) x) \rightarrow_\alpha ((\lambda z : \phi . y) x)$$
А терм $\lambda x : \phi . y$, например, не находится в отношении $\alpha$-конверсии с термом
$\lambda y : \phi . y$, так свободная переменная $y$ становится связанной. Также как и
терм $(\lambda x_1 : \phi . x_1)(\lambda y_1 : \psi . y_1)$ не находится в отношении $\alpha$-конверсии
с термом $(\lambda x_2 : \phi . x_2)(\lambda y_2 : \psi . y_2)$, так как
здесь происходит переименование переменных в двух подтермах одновременно (однако данные термы
находятся в транзитивном отношении $\twoheadrightarrow_\alpha$).

Во-вторых, мы свяжем $\lambda$-абстракции с функциями от связанных переменных,
а $\lambda$-применения \dash с вычислениями функций на аргументах.
Пусть в терме $M$ есть подтерм $N = (\lambda x : \phi . L)K$. Тогда $M$ 
находится в отношении  $\beta$-редукции $M \rightarrow_\beta M'$ с некоторым термом $M'$, 
получаемым из $M$ заменой какого-либо одного вхождения подтерма $N$ на подтерм
$L[K/x]$ при условии корректности данной подстановки: никакая свободная переменная $K$ не
становится после такой подстановки связанной. Приведем несколько примеров:
$$(\lambda x : \phi . x) y \rightarrow_\beta y$$
$$(\lambda y : \phi . y) y \rightarrow_\beta y$$
$$(\lambda x : \phi . y)((\lambda y : \psi . y) z) \rightarrow_\beta (\lambda x : \phi . y) z \rightarrow_\beta y$$

Таким образом, положив отношение $\rightarrow$ равным объединению $\rightarrow_\alpha$ и $\rightarrow_\beta$,
получаем формальную систему переписывания термов $(\Lambda Terms, \rightarrow_{\alpha \beta}, \twoheadrightarrow_{\alpha \beta})$.

\begin{definition}
    \label{NormalFormDefinition}
    Пусть $(A, \rightarrow, \twoheadrightarrow)$ \dash формальная система переписывания термов.
    Элемент $a \in A$ является нормальной формой, если $\forall b \in A (a \not \rightarrow b)$.
\end{definition}

Общее определение \ref{NormalFormDefinition}, однако, не очень хорошо работает
для нашей системы $(\Lambda Terms, \rightarrow_{\alpha \beta}, \twoheadrightarrow_{\alpha \beta})$:
к любому замкнутому терму $M$ можно применить $\alpha$-конверсию. Поэтому мы будем говорить 
о нормальной форме с точностью до $\alpha$-конверсии: терм $M$ находится в нормальной форме
если $\forall N \in \Lambda Term (M \rightarrow_{\alpha \beta} N \Rightarrow M \rightarrow_\alpha N)$.

\begin{definition}
    \label{NormalizationDefinition}
    Элемент $a$ системы $(A, \rightarrow, \twoheadrightarrow)$ называется (слабо-)нормализуемым, если
    найдется такой $b \in A$, что 
    \begin{enumerate}
        \item $b$ является нормальной формой.
        \item $a \twoheadrightarrow b$.
    \end{enumerate}
    (В частности, всякая нормальная форма является нормализуемой, в силу рефлексивности отношения
    $\twoheadrightarrow$.)
\end{definition}

\begin{definition}
    \label{StrongNormalizationDefinition}
    Элемент $a$ системы $(A, \rightarrow, \twoheadrightarrow)$ называется сильно-нормализуемым,
    если
    \begin{enumerate}
        \item $a$ является нормализуемым.
        \item Не верно следующие утверждение: существует счетная последовательность
        элементов $A$ $\{ b_0, b_1, b_2, \dots \}$ такая, что $b_0 = a$ и 
        $b_i \rightarrow b_{i+1}$ для любого натурального $i$.
    \end{enumerate}
\end{definition}
Иными словами, определение \ref{StrongNormalizationDefinition} говорит, что $a$ является
сильно-нормализуемым, если всякая последовательность преобразований $\rightarrow$ элемента $a$ приводит
к нормальной форме за конечное число шагов.

Проиллюстрируем определения \ref{NormalFormDefinition} и \ref{StrongNormalizationDefinition} примерами из
нашего $\lambda$-исчисления. Терм $(\lambda x : p . y)((\lambda z : q . z) w)$ является сильно нормализуемым:
мы либо можем переписать внутренний подтерм $(\lambda z : q) w$ с помощью $\beta$-редукции и получить
терм $(x : p . y) w$, который единственным образом $\beta$-редуцируется до нормальной формы $y$, либо
либо сразу переписать самое внешние $\lambda$-применение до $y$. Других последовательностей $\beta$-редукций
для такого терма нет. С другой стороны, терм $(\lambda x : p . y)((\lambda z : q . zz)(\lambda z : q . zz))$
является только слабо-нормализуемым: как и в прошлом примере, мы можем сразу средуцироваться до $y$,
или попытаться переписывать подтерм $(\lambda z : q . zz)(\lambda z : q . zz)$ и получить бесконечную последовательность
$$(\lambda x : p . y)((\lambda z : q . zz)(\lambda z : q . zz)) \rightarrow_\beta (\lambda x : p . y)((\lambda z : q . zz)(\lambda z : q . zz)) \rightarrow_\beta \dots$$
Сразу же замечаем, что терм $(\lambda z : q . zz)(\lambda z : q . zz)$ не является ни сильно-,
ни слабо-нормализуемым.

\begin{definition}
    \label{ChurchRosserDefinition}
    Система $(A, \rightarrow, \twoheadrightarrow)$ обладает свойством Чёрча-Россера (свойством конфлюэнтности),
    если 
    $$\forall a, b, b' \in A \Bigl((a \twoheadrightarrow b \land a \twoheadrightarrow b') \Rightarrow \exists c \in A (b \twoheadrightarrow c \land b' \twoheadrightarrow c) \Bigr)$$
    $$
    \begin{tikzcd}
        & \arrow[ld, two heads] a \arrow[rd, two heads] & \\
        b \arrow[rd, two heads, dashed] & & \arrow[ld, two heads, dashed] b' \\
        & c &
    \end{tikzcd}
    $$
\end{definition}

\begin{lemma}
    \label{UniqueNormalFormLemma}
    Если для системы $(A, \rightarrow, \twoheadrightarrow)$ выполняется свойство Чёрча-Россера, и
    $a$ \dash нормализуемый элемент $A$, то его нормальная форма $b$ единственна.
\end{lemma}
\begin{proof}
    Пусть $a$ редуцируется к двум нормальным формам $b$ и $b'$. Тогда по свойству
    конфлюэнтности найдется такой $c$, что $b \twoheadrightarrow c$ и $b' \twoheadrightarrow c$. Но
    любая нормальная форма находится в отношении $\twoheadrightarrow$ только сама с собой.
    А значит
    $$b = c = b'$$
\end{proof}

\begin{theorem}[\cite{SorUrz06}]
    Система $(\Lambda Terms, \rightarrow_{\alpha \beta}, \twoheadrightarrow_{\alpha \beta})$ обладает
    свойством Чёрча-Россера.
\end{theorem}

Существует несколько способов закодировать натуральные числа в $\lambda$-исчислении. Один из таких
способов \dash построить для каждого натурального $n$ комбинатор $[n]$, называемый нумералом Чёрча,
следующим образом:
$$[0] := \lambda f : p \rightarrow p . \lambda x : p . x$$
$$[1] := \lambda f : p \rightarrow p . \lambda x : p . fx$$
$$[2] := \lambda f : p \rightarrow p . \lambda x : p . f (fx)$$
$$[3] := \lambda f : p \rightarrow p . \lambda x : p . f (f (fx))$$
$$ \dots $$
$$[n] := \lambda f : p \rightarrow p. \lambda x : p . \underbrace{f(\dots (f}_{n\ \text{применений}\ f} x))$$
$$ \dots $$

\begin{theorem}[\cite{SorUrz06}]
    Для всякой вычислимой функции $f$ из $\mathbb{N}$ в $\mathbb{N}$ существует
    $\lambda$-терм $F$ такой, что
    \begin{enumerate}
        \item Если $n$ не принадлежит области определения $f$, то терм $F[n]$ не нормализуем.
        \item Если $n$ принадлежит области определения $f$, то $F[n] \twoheadrightarrow_{\alpha \beta} [f(n)]$.
    \end{enumerate}
\end{theorem}

По лемме \ref{UniqueNormalFormLemma} всякий нумерал Чёрча единственен с точностью до переименования
связанных переменных, а значит редукция термов $F[n]$ задаёт корректною функциональное отношение на
$\mathbb{N}$. Таким образом система $(\Lambda Terms, \rightarrow_{\alpha \beta}, \twoheadrightarrow_{\alpha \beta})$
является полноценной, полной по Тьюрингу моделью вычислений.

Однако, до сих в пор в нашем языке $\lambda$-исчисления остаются термы вида $xx$, которым невозможно приписать
какое-либо вычислительный смысл. Мы "отфильтруем" такие "плохие" термы, введя дедуктивную систему о типах:
мы будем делить типы на те, для которых можно вывести суждение о наличие у них некоторого типа (типизуемый терм),
и те, у которых такое суждение вывести нельзя (нетипизуемые термы).

\begin{definition}
    \label{ContextDefinition}
    Контекст $\Gamma$ \dash это конечное, возможно пустое, множество пар вида $x : \phi$, где $x$ \dash $\lambda$-переменная,
    а $\phi$ \dash некоторый тип ($\phi \in Types$), причем для любых двух пар $(a : \phi)$ и $(b : \psi)$ из
    $\Gamma$ верно
    $$a = b \Rightarrow \phi = \psi$$
    Запись $\Gamma, x : \phi$ означает контекст $\Gamma \cup \{x: \phi\}$.
\end{definition}

\begin{definition}
    \label{JudgmentDefinition}
    Суждение \dash это формальное утверждение вида
    $$\Gamma \vdash M : \phi$$
    которое читается как "в контексте $\Gamma$ терм $M$ имеет тип $\phi$".

    Если $\Gamma = \emptyset$, то вместо 
    $$\emptyset \vdash M : \phi$$ 
    пишут просто
    $$\vdash M : \phi$$

    Вывод суждение $\Gamma \vdash M : \phi$ \dash это конечное дерево, чьи вершины помечены суждениями,
    корень помечен суждением $\Gamma \vdash M : \phi$, 
    листья помечены аксиомами (правилами без посылок)
    \begin{prooftree}
        \AxiomC{}
            \RightLabel{$Var$}
        \UnaryInfC{$\Gamma, x : \phi \vdash x : \phi$}
    \end{prooftree}
    а ребрам соответствуют правила вывода, описывающие способы получения одних суждений из других:
    \begin{multicols}{2}
    \begin{prooftree}
        \AxiomC{$\Gamma, x : \phi \vdash M : \psi$}
            \RightLabel{$Abs$}
        \UnaryInfC{$\Gamma \vdash (\lambda x : \phi . M) : \phi \rightarrow \psi$}
    \end{prooftree}
    \begin{prooftree}
        \AxiomC{$\Gamma \vdash M : \phi \rightarrow \psi$}
        \AxiomC{$\Gamma \vdash N : \phi$}
            \RightLabel{$App$}
        \BinaryInfC{$\Gamma \vdash (MN) : \psi$}
    \end{prooftree}
    \end{multicols}
    Во всех правилах вывода, $\Gamma$ \dash произвольный контекст.
\end{definition}

Приведем несколько примеров выводов суждений. Для начала, выведем суждение 
$x : p \rightarrow q, y : p \vdash (xy) : q$
\begin{prooftree}
    \AxiomC{}
        \RightLabel{$Var$}
    \UnaryInfC{$ x : p \rightarrow q, y : p \vdash x : p \rightarrow q$}
    \AxiomC{}
        \RightLabel{$Var$}
    \UnaryInfC{$ x : p \rightarrow q, y : p \vdash y : p$}
        \RightLabel{$App$}
    \BinaryInfC{$ x : p \rightarrow q, y : p \vdash (xy) : q$}
\end{prooftree}
Другой пример: $\vdash (\lambda x : p . x) : p \rightarrow p$
\begin{prooftree}
    \AxiomC{}
        \RightLabel{$Var$}
    \UnaryInfC{$ x : p \vdash x : p$}
        \RightLabel{$Abs$}
    \UnaryInfC{$\vdash (\lambda x : p . x) : p \rightarrow p$}
\end{prooftree}
Ну и третий пример: $\vdash (\lambda f : p \rightarrow p . \lambda x : p . f (f x)) : (p \rightarrow p) \rightarrow p \rightarrow p$
\begin{prooftree}
    \AxiomC{}
        \RightLabel{$Var$}
    \UnaryInfC{$f : p \rightarrow p, x : p \vdash f : p \rightarrow p$}
    \AxiomC{}
        \RightLabel{$Var$}
    \UnaryInfC{$f : p \rightarrow p, x : p \vdash f : p \rightarrow p$}
    \AxiomC{}
        \RightLabel{$Var$}
    \UnaryInfC{$f : p \rightarrow p, x : p \vdash x : p$}
        \RightLabel{$App$}
    \BinaryInfC{$f : p \rightarrow p, x : p \vdash (f x) : p$}
        \RightLabel{$App$}
    \BinaryInfC{$f : p \rightarrow p, x : p \vdash (f (f x)) : p$}
        \RightLabel{$Abs$}
    \UnaryInfC{$f : p \rightarrow p \vdash (\lambda x : p . f (f x)) : p \rightarrow p$}
        \RightLabel{$Abs$}
    \UnaryInfC{$\vdash (\lambda f : p \rightarrow p . \lambda x : p . f (f x)) : (p \rightarrow p) \rightarrow p \rightarrow p$}
\end{prooftree}

Естественно, система переписывание термов $(\Lambda Terms, \rightarrow_{\alpha \beta}, \twoheadrightarrow_{\alpha \beta})$
согласованна с дедуктивной системой вывода суждений о типизации.
\begin{lemma}[\cite{SorUrz06}]
    Пусть $M$ \dash $\lambda$-терм такой, что
    \begin{enumerate}
        \item Суждение $\Gamma \vdash M : \phi$ выводимо для некоторого контекста $\Gamma$
        и некоторого типа $\phi$.
        \item $M \twoheadrightarrow_{\alpha \beta} N$ для некоторого терма $N$.
    \end{enumerate}
    Тогда суждение $\Gamma \vdash N : \phi$ также выводимо.
\end{lemma}

Систему переписываний $(\Lambda Terms, \rightarrow_{\alpha \beta}, \twoheadrightarrow_{\alpha \beta})$
вместе с дедуктивной системой из определения \ref{JudgmentDefinition} принято называть
простым типизованным по Чёрчу $\lambda$-исчислением и обозначать $\lambda_\rightarrow$.

\begin{theorem}
    \label{SimplyTypedStrongNormaliztionTheorem}
    Пусть $M$ \dash типизуемый в некотором контексте $\Gamma$ терм (для него выводится суждение $\Gamma \vdash M : \phi$ для некоторого типа $\phi$).
    Тогда $M$ является сильно-нормализуемым.
\end{theorem}

Следствием теоремы \ref{SimplyTypedStrongNormaliztionTheorem} является тот факт, что типизуемые термы
больше не являются полной по Тьюрингу моделью вычислений. Вместо этого типизуемым термам
соответствует некоторое собственное подмножество тотальных вычислимых функций, называемое расширенными
многочленами \cite{Zakr07}: наименьшее множество вычислимых функций,
замкнутое относительно взятия композиции, содержащие
\begin{enumerate}
    \item Проекции из кортежей натуральных чисел.
    \item Константные функции.
    \item Сложение.
    \item Умножение.
    \item Функцию $ifzero(n, m, p)$: если $n = 0$, то $m$, в противном случае $p$.
\end{enumerate}

Однако, типизуемые термы имеют не только вычислительную интерпретацию,
но и устанавливают соответствие между $\lambda$-исчислением и логикой.

\begin{theorem}[Соответствие Карри \dash Ховарда \cite{SorUrz06}]
    \label{CarryHowardTheorem}
    В исчислении $\lambda_\rightarrow$ суждение $\Gamma \vdash N : \phi$ выводится тогда и только
    тогда, когда $\operatorname{range}(\Gamma) \vdash \phi$ верно для импликативного фрагмента логики высказываний $IPC$,
    где $\operatorname{range}$ \dash функция из множества пар переменных и типов в множество формул логики высказываний:
    $$\operatorname{range} : (x : \psi) \mapsto \psi$$
\end{theorem}

Соответствие Карри \dash Ховарда дает старт парадигме "утверждения-как-типы": мы берем некоторое типизованное
$\lambda$-исчисление, интерпретируем типы как формальные утверждения (теоремы), а соответствующие
$\lambda$-термы \dash как доказательства. Причем такой подход заведома конструктивен: $\lambda$-исчисление
продолжает сохранять некоторую вычислительную интерпретацию, а значит термы-доказательства
являются конструкциями в стиле интерпретации Брауера \dash Гейтинга \dash Колмогорова.

Усложняя язык $\lambda$-термов и типов и расширяя правила типизации можно получать все более выразительные
исчисления, которые, с одной стороны, устанавливают соответствия со все более содержательными логиками,
а с другой стороны, типизуют все большие подмножества тотальных вычислимых функций,
сохраняя все хорошие мета-свойства, такие как сильная нормализуемость и конфлюэнтность. Двигаясь по первому пути,
можно пополнить язык термов конструкциями прямой суммы и прямого произведения, а также константой $\bot$,
установив соответствие между $\lambda$-исчислением и интуиционистской логикой высказываний $Int$.
А разрешив определять типы через термы, и введя тем самым так называемые зависимые типы, можно
получить исчисление $\lambda P_1$, которому соответствует интуиционистская логика предикатов.

Двигаясь же по второму пути, мы можем разрешить навешивать кванторы на типы, получив так называемую Систему $F$.
Чуть более формально, язык системы $F$ позволяет писать термы вида 
$\Lambda p . M$
и
$M \phi$
где $\alpha$ \dash это тип. Иными словами, в системе $F$ разрешены не только $\lambda$-абстракции
и $\lambda$-применения по термам, но и по типам. Соответствующая грамматика термов принимает вид
$$t := x | (\lambda x : \phi . t) | (tt) | (\Lambda p . t) | (t \phi)$$
($x$ \dash $\lambda$-переменная, $p$ \dash типовая переменная, а $\phi$ \dash тип). Система переписывания термов
пополняется отношением $\beta'$-редукции
$$(\Lambda \alpha . M) \phi \rightarrow_{\beta'} M[\phi/\alpha]$$
Пример:
$$(\Lambda \alpha . \lambda f : \alpha \rightarrow \alpha . \lambda x : \alpha . f x) p \rightarrow_{\beta'} \lambda f : p \rightarrow p . \lambda x : p . f x$$
В типах появляется квантор всеобщности по типовым переменным
$$type := p | (type \rightarrow type) | (\forall p . type)$$
благодаря чему запись $\forall . p (p \rightarrow p)$, например, становится валидным типом.
Дедуктивная система вывода суждений о типизации Системы $F$ имеет следующий набор правил:
\begin{multicols}{2}
    \begin{prooftree}
        \AxiomC{}
            \RightLabel{$Var$}
        \UnaryInfC{$\Gamma, x : \phi \vdash x : \phi$}
    \end{prooftree}
    \begin{prooftree}
        \AxiomC{$\Gamma, x : \phi \vdash M : \psi$}
            \RightLabel{$Abs$}
        \UnaryInfC{$\Gamma \vdash (\lambda x : \phi . M) : \phi \rightarrow \psi$}
    \end{prooftree}
\end{multicols}
\begin{multicols}{2}
    \begin{prooftree}
        \AxiomC{$\Gamma \vdash M : \phi \rightarrow \psi$}
        \AxiomC{$\Gamma \vdash N : \phi$}
            \RightLabel{$App$}
        \BinaryInfC{$\Gamma \vdash (MN) : \psi$}
    \end{prooftree}
    \begin{prooftree}
        \AxiomC{$\Gamma \vdash M : \forall p.\phi$}
            \RightLabel{$Inst$}
        \UnaryInfC{$\Gamma \vdash (M \psi) : \phi[\psi/p]$}
    \end{prooftree}
\end{multicols}
\begin{prooftree}
    \AxiomC{$\Gamma \vdash M : \phi$}
        \RightLabel{(если $p \not \in FVar(\Gamma)$) $Gen$}
    \UnaryInfC{$\Gamma \vdash (\Lambda p . M) : \forall p . \phi$}
\end{prooftree}

Типизуемые термы Системы $F$ продолжают быть сильно-нормализуемыми\cite{SorUrz06}, однако им 
соответствует гораздо большее подмножество вычислимых функций, чем для типизуемых термов $\lambda_\rightarrow$.

\begin{theorem}[\cite{Gira71}]
    Пусть функция $g : \mathbb N \rightarrow \mathbb N$ определима в арифметике Пеано второго порядка.
    Тогда в Системе $F$ существует такой типизуемый терм $G$, что
    $$G[n]_F \rightarrow_{\alpha \beta \beta'} [g(n)]_F$$
    для любого $n \in \mathbb N$ (здесь $[n]_F$ \dash нумерал для числа $n$ в Системе $F$).
\end{theorem}

\subsection{Теория типов Мартин-Лёфа}

Интуиционистская теория типов Мартин-Лёфа $MLTT$ является типизованным $\lambda$-исчислением, в котором,
с одной стороны, типизуем довольно большой класс вычислимых функций, а с другой стороны, соответствующий
язык термов типов достаточно богат, чтобы претендовать на формальные основания конструктивной
математики в парадигме "утверждения как типы". $MLTT$ не соответствует напрямую никакой интуиционистской логике
(в смысле соответствия Карри \dash Ховарда), являясь самостоятельной формальной системой.

Язык $\lambda$-термов $MLTT$, по мимо стандартных конструкций простого $\lambda$-исчисления, содержит
примитивные константы $c, c', \dots$ и функциональные константы $f, f', \dots$, которые мы будем вводить
далее по ходу повествования, а также специальные $Pi$- и $\Sigma$-абстракции. Грамматика термов $MLTT$ имеет вид
$$t := \; x \; | \; \lambda x : t . t \; | \; t(t) \; | \; \prod_{(x : t)} t \; | \; \sum_{(x : t)} t \; | \; c \; | \; f$$
Здесь $t(t)$ \dash запись $\lambda$-аппликации в более привычном математическом виде применения функции к аргументу,
такая нотация является более подходящей для $MLTT$ из-за усложнения синтаксиса.

Язык термов не определяется отдельно, вместо этого мы считаем, что язык типов совпадает с языком термов.
Это означает, что у нас появляются специальные термы, которые, в каком-то смысле, являются именами типов.
Такие термы-имена мы будем обозначать теперь заглавными латинскими буквами $A, A', B, \dots$.
Первым примером такого рода термов является счетное семейство примитивных констант
$$\mathcal U_0, \mathcal U_1, \dots, \mathcal U_i, \dots$$
Константа $\mathcal U_i$ означает универсум типов уровня $i$, а запись $A : \mathcal U_i$ означает,
что терм $A$ является именем типа (или просто типом) из $i$-ого универсума.

Множество свободных переменных терма все также определяется рекурсивно:
\begin{enumerate}
    \item $FVar(x) = \{ x \}$
    \item $FVar(c) = \emptyset$
    \item $FVar(f) = \emptyset$
    \item $FVar(a(b)) = FVar(a) \cup FVar(b)$
    \item $FVar(\lambda x : A . b) = FVar(A) \cup (FVar(b) \setminus \{ x \})$
    \item $FVar(\prod \limits_{x : A} B) = FVar(A) \cup (FVar(B) \setminus \{ x \})$
    \item $FVar(\sum \limits_{x : A} B) = FVar(A) \cup (FVar(B) \setminus \{ x \})$
\end{enumerate}

В отличие от $\lambda_\rightarrow$, мы не будем раздельно определять понятие контекста, структуры формальной
системы переписывания термов и исчисление суждений о типизации, а объединим всё это
в одну большую дедуктивную систему, похожую на исчисление естественного вывода. В этой системе есть
три вида суждений:
\begin{enumerate}
    \item $\Gamma \; ctx$ \dash контекст $\Gamma$ корректно определён.
    \item $\Gamma \vdash a : A$ \dash в контексте $\Gamma$ терм $a$ имеет тип $A$.
    \item $\Gamma \vdash a \doteq b : A$ \dash в контексте $\Gamma$ $a$ и $b$ являются
    эквивалентными термами типа $A$.
\end{enumerate}
Далее мы перечислим правила вывода, позволяющие получать подобные суждения.

Во-первых, мы хотим считать пустой контекст корректным, и выражаем это через правило
\begin{prooftree}
    \AxiomC{}
        \RightLabel{$ctx-EMP$}
    \UnaryInfC{$\cdot  \; ctx$}
\end{prooftree}
Далее мы описываем способ получения корректного контекста пополнением парами $x : A$
\begin{prooftree}
    \AxiomC{$x_1 : A_1, \dots, x_n : A_n \vdash A_{n+1} : \mathcal U_i$}
        \RightLabel{$ctx-EXT$}
    \UnaryInfC{$(x_1 : A_1, \dots, x_n : A_n, x_{n+1} : A_{n+1}) \; ctx$}
\end{prooftree}
Заметим, что теперь контекст \dash это упорядоченная конечная последовательность пар: тип $A_n$ переменной $x_n$
сам является термом, и может зависеть от свободных переменных, но только тех, которое до этого
уже были добавлены в контекст.

Введем аналог правила $Var$ исчисления $\lambda_\rightarrow$ для $MLTT$:
\begin{prooftree}
    \AxiomC{$(x_1 : A_n, \dots, x_i : A_i, \dots, x_n : A_n) \; ctx$}
    \RightLabel{$Var$}
    \UnaryInfC{$x_1 : A_n, \dots, x_i : A_i, \dots, x_n : A_n \vdash x_i : A_i$}
\end{prooftree}

Теперь введем правила, задающие полиморфизм универсумов типов $\mathcal U_i$.
\begin{multicols}{2}
    \begin{prooftree}
        \AxiomC{$\Gamma \; ctx$}
            \RightLabel{$\mathcal U-INTRO$}
        \UnaryInfC{$\Gamma \vdash \mathcal U_i :\mathcal U_{i+1}$}
    \end{prooftree}
    \begin{prooftree}
        \AxiomC{$\Gamma \vdash A : \mathcal U_i$}
            \RightLabel{$\mathcal U-CUMUL$}
        \UnaryInfC{$\Gamma \vdash A : \mathcal U_{i+1}$}
    \end{prooftree}
\end{multicols}
Правило $\mathcal U-INTRO$ в частности гласит, что каждая примитивная константа $\mathcal U_i$ сама является
типом в смысле замечания из начала данного раздела.

Далее, зададим базовые свойства отношения эквивалентности термов.
\begin{multicols}{2}
    \begin{prooftree}
        \AxiomC{$\Gamma \vdash a : A$}
        \UnaryInfC{$\Gamma \vdash a \doteq a : A$}
    \end{prooftree}
    \begin{prooftree}
        \AxiomC{$\Gamma \vdash a \doteq b : A$}
        \UnaryInfC{$\Gamma \vdash b \doteq a : A$}
    \end{prooftree}
\end{multicols}
\begin{prooftree}
        \AxiomC{$\Gamma \vdash a \doteq b : A$}
        \AxiomC{$\Gamma \vdash b \doteq c : A$}
        \BinaryInfC{$\Gamma \vdash a \doteq c : A$}
\end{prooftree}
\begin{multicols}{2}
    \begin{prooftree}
        \AxiomC{$\Gamma \vdash a : A$}
        \AxiomC{$\Gamma \vdash A \doteq B : \mathcal U_i$}
        \BinaryInfC{$\Gamma \vdash a : B$}
    \end{prooftree}
    \begin{prooftree}
        \AxiomC{$\Gamma \vdash a \doteq b : A$}
        \AxiomC{$\Gamma \vdash A \doteq B : \mathcal U_i$}
        \BinaryInfC{$\Gamma \vdash a \doteq b : B$}
    \end{prooftree}
\end{multicols}

Следующий набор правил принято называть структурой теории типов, так как они задают свойства контекстов
и подстановок. Во-первых, имеем два правила ослабления, которые говорят, что выводимость суждений
сохраняется при расширениях контекстов:
\begin{multicols}{2}
    \begin{prooftree}
        \AxiomC{$\Gamma \vdash A : \mathcal U_i$}
        \AxiomC{$\Gamma, \Delta \vdash b : B$}
            \RightLabel{$Wkg_1$}
        \BinaryInfC{$\Gamma, x : A, \Delta \vdash b : B$}
    \end{prooftree}
    \begin{prooftree}
        \AxiomC{$\Gamma \vdash A : \mathcal U_i$}
        \AxiomC{$\Gamma, \Delta \vdash a \doteq b : B$}
            \RightLabel{$Wkg_2$}
        \BinaryInfC{$\Gamma, x : A, \Delta \vdash a \doteq b : B$}
    \end{prooftree}
\end{multicols}
Во-вторых, имеем правила подстановок:
\begin{multicols}{2}
    \begin{prooftree}
        \AxiomC{$\Gamma \vdash a : A$}
        \AxiomC{$\Gamma, x : A, \Delta \vdash b : B$}
            \RightLabel{$Subst_1$}
        \BinaryInfC{$\Gamma, \Delta[a/x] \vdash b[a/x] : B[a/x]$}
    \end{prooftree}
    \begin{prooftree}
        \AxiomC{$\Gamma \vdash a : A$}
        \AxiomC{$\Gamma, x : A, \Delta \vdash b \doteq c : B$}
            \RightLabel{$Subst_2$}
        \BinaryInfC{$\Gamma, \Delta[a/x] \vdash b[a/x] \doteq c[a/x] : B[a/x]$}
    \end{prooftree}
\end{multicols}
\begin{prooftree}
    \AxiomC{$\Gamma \vdash a \doteq b : A$}
    \AxiomC{$\Gamma, x : A, \Delta \vdash c : C$}
        \RightLabel{$Subst_3$}
    \BinaryInfC{$\Gamma, \Delta[a/x] \vdash c[a/x] \doteq c[b/x] : C[a/x]$}
\end{prooftree}
Естественно, во всех правилах $Subst_i$ подстановка должна быть корректной: никакая переменная, которая до
подстановки была свободной, не должна стать связанной. 

Теперь мы можем перейти к обсуждению содержательных типов. $\Pi$-тип, также известный как тип зависимого
произведения, является обобщением стрелочного типа $\phi \rightarrow \psi$ из системы $\lambda_\rightarrow$.
По правилу $Abs$, если в предположении, что переменная $x$ имеет тип $A$, мы можем заключить, что терм $t$
имеет тип $B$, то мы можем с помощью $\lambda$-абстракции получить тип $A \rightarrow B$. Однако в $MLTT$
тип $B$ сам является термом, а значит сам может зависеть от свободной переменной $x$, мы должны учесть это
в наших правилах вывода. Для этого мы вводим синтаксическую конструкцию $\Pi$-абстракции и правило формирование типа
зависимого произведения:
\begin{prooftree}
    \AxiomC{$\Gamma \vdash A : \mathcal U_i$}
    \AxiomC{$\Gamma, x : A \vdash b : B$}
        \RightLabel{$\Pi-FORM$}
    \BinaryInfC{$\Gamma \vdash (\prod \limits_{x:A} B) : \mathcal U_i$}
\end{prooftree}

Затем мы вводим правило введение $\Pi$-типа, которое описывает способ получение термов типа зависимого
произведение (аналог правила $Abs$):
\begin{prooftree}
    \AxiomC{$\Gamma, x : A \vdash b : B$}
        \RightLabel{$\Pi-INTRO$}
    \UnaryInfC{$\Gamma \vdash (\lambda x : A . b) : \prod \limits_{x : A} B$}
\end{prooftree}
Причем, как и в случае $\lambda_\rightarrow$, $\lambda$-абстракция связывает свободные вхождения переменной
$x$ в терм  $b$.

Далее, введем правило элиминации, описывающие способ получения новых типов, если у нас уже есть терма
типа зависимого произведения.
\begin{prooftree}
    \AxiomC{$\Gamma \vdash f : \prod \limits_{x : A} B$}
    \AxiomC{$\Gamma \vdash a : A$}
        \RightLabel{$\Pi-ELIM$}
    \BinaryInfC{$\Gamma \vdash f(a) : B[a/x]$}
\end{prooftree}

Наконец, введем вычислительное правило, описывающее класс эквивалентности термов $\Pi$-типа (аналог 
рефлексивно-транзитивного замыкания отношения $\beta$-редукции):
\begin{prooftree}
    \AxiomC{$\Gamma, x : A \vdash b : B$}
    \AxiomC{$\Gamma \vdash a : A$}
        \RightLabel{$\Pi-COMP$}
    \BinaryInfC{$\Gamma \vdash (\lambda x : A . b)(a) \doteq b[a/x] : B[a/x]$}
\end{prooftree}
Причем подстановка в правиле $\Pi-COMP$, как и раньше, должна быть корректной.

Следующий набор правил устанавливает взаимодействие $\Pi$-типов с отношением $\doteq$:
\begin{prooftree}
    \AxiomC{$\Gamma \vdash A : \mathcal U_I$}
    \AxiomC{$\Gamma, x : A \vdash B \doteq B' : \mathcal U_i$}
        \RightLabel{$\Pi-FORM-EQ$}
    \BinaryInfC{$\Gamma \vdash (\prod \limits_{x : A} B) \doteq (\prod \limits_{x : A} B') : \mathcal U_i$}
\end{prooftree}
\begin{prooftree}
    \AxiomC{$\Gamma, x : A \vdash b \doteq b' : B$}
        \RightLabel{$\Pi-INTRO-EQ$}
    \UnaryInfC{$\Gamma \vdash (\lambda x : A . b) \doteq (\lambda x : A . b') : \prod \limits_{x : A} B$}
\end{prooftree}
\begin{prooftree}
    \AxiomC{$\Gamma \vdash f : \prod \limits_{x : A} B$}
    \AxiomC{$\Gamma \vdash a \doteq a' : A$}
        \RightLabel{$\Pi-ELIM-EQ$}
    \BinaryInfC{$\Gamma \vdash f(a) \doteq f(a') : B[a/x]$}
\end{prooftree}

С аналогами правил $\Pi-INTRO$ и $\Pi-ELIM$ мы уже сталкивались, рассмотрим пример вывода суждения
$\vdash \prod_{(x : \mathcal U_0)} \mathcal U_0 : \mathcal U_1$ c использованием нового правила $\Pi-FORM$:
\begin{prooftree}
    \AxiomC{}
        \RightLabel{$ctx-EMP$}
    \UnaryInfC{$\cdot \; ctx$}
        \RightLabel{$\mathcal U - INTRO$}
    \UnaryInfC{$\vdash \mathcal U_0 : \mathcal U_1$}
    \AxiomC{}
        \RightLabel{$ctx-EMP$}
    \UnaryInfC{$\cdot \; ctx$}
        \RightLabel{$\mathcal U - INTRO$}
    \UnaryInfC{$\vdash \mathcal U_0 : \mathcal U_1$}
        \RightLabel{$ctx-EXT$}
    \UnaryInfC{$(x : \mathcal U_0) \; ctx$}
        \RightLabel{$\mathcal U - INTRO$}
    \UnaryInfC{$x : \mathcal U_0 \vdash \mathcal U_0 : \mathcal U_1$}
        \RightLabel{$\Pi-FORM$}
    \BinaryInfC{$\vdash (\prod \limits_{x : \mathcal U_0} \mathcal U_0) : \mathcal U_1$}
\end{prooftree}
Другой пример, вывод суждения $\vdash (\lambda x : \mathcal U_3 . x)(\mathcal U_2) \doteq \mathcal U_2 : \mathcal U_3$,
с помощью правила $\Pi-COMP$ (в этот раз для лучшей читаемости мы не будем писать имя каждого правила вывода):
\begin{prooftree}
    \AxiomC{}
    \UnaryInfC{$\cdot \; ctx$}
    \UnaryInfC{$\vdash \mathcal U_3 : \mathcal U_4$}
    \UnaryInfC{$(x : \mathcal U_3) \; ctx$}
    \UnaryInfC{$x : \mathcal U_3 \vdash x : \mathcal U_3$}
    \AxiomC{}
    \UnaryInfC{$\cdot \; ctx$}
    \UnaryInfC{$\vdash \mathcal U_2 : \mathcal U_3$}
    \BinaryInfC{$\vdash (\lambda x : \mathcal U_3 . x)(\mathcal U_2) \doteq \mathcal U_2 : \mathcal U_3$}
\end{prooftree}

С точки зрения формализации математики, типам зависимого произведения соответствуют одновременно квантор
всеобщности и импликация в смысле интерпретации Брауэра \dash Гейтинга \dash Колмогорова: если
зависимый тип $B(a)$ выражает некоторое свойство терма $a$, то тип $\prod_{(x : A)} B$ выражает
способ получить терм-доказательство свойства $B(x)$ для любого $x$ из типа $A$.

Введем теперь семейство индуктивных типов. Каждый индуктивный тип определяется правилами
формации, введения, элиминации и вычисления, а также правилами $FORM-EQ$, $INTRO-EQ$ и $ELIM-EQ$,
по аналогии с правилами $\Pi-FORM-EQ$, $\Pi-INTRO-EQ$ и $\Pi-ELIM-EQ$ соответственно
(мы не будем выписывать такие правило явно из соображения экономии, в силу их простого устройства).
Начнем с типа $\mathbb 1$, населенного единственной примитивной константой $*$.
\begin{multicols}{2}
    \begin{prooftree}
        \AxiomC{$\Gamma \; ctx$}
            \RightLabel{$\mathbb 1-INTRO$}
        \UnaryInfC{$\Gamma \vdash \mathbb 1 : \mathcal U_0$}
    \end{prooftree}
    \begin{prooftree}
        \AxiomC{$\Gamma \; ctx$}
            \RightLabel{$\mathbb 1-INTRO$}
        \UnaryInfC{$\Gamma \vdash * : \mathbb 1$}
    \end{prooftree}
\end{multicols}
    \begin{prooftree}
        \AxiomC{$\Gamma, x : \mathbb 1 \vdash C:\mathcal U_i$}
        \AxiomC{$\Gamma \vdash c : C[*/x]$}
        \AxiomC{$\Gamma \vdash a : \mathbb 1$}
            \RightLabel{$\mathbb 1-ELIM$}
        \TrinaryInfC{$\Gamma \vdash \ind_{\mathbb 1}(x.C, c, a) : C[a/x]$}
    \end{prooftree}    
    \begin{prooftree}
        \AxiomC{$\Gamma, x : \mathbb 1 \vdash C : \mathcal U_i$}
        \AxiomC{$\Gamma \vdash c : C[*/x]$}
            \RightLabel{$\mathbb 1-COMP$}
        \BinaryInfC{$\Gamma \vdash \ind_{\mathbb 1}(x.C, c, *) \doteq c : C[*/x]$}
    \end{prooftree}

Правила $\mathbb 1-ELIM$ и $\mathbb 1-COMP$ вводят функциональную константу $\ind_{\mathbb 1}$, называемую
индуктором, которая, с одной стороны, позволяет получить тип $C(x)$ для любого терма $x$ типа $\mathbb 1$,
в предположение, что мы можем построить терм $c$ конкретного типа $C[*/x]$, а с другой стороны, позволяет
задавать тотальные рекурсивные функции от термов типа $\mathbb 1$. Запись $x.C$ является сокращением
для терма $(\lambda x : \mathbb 1 . C) : \prod_{(x : \mathbb 1)} \mathcal U_i$, и подчеркивает, что
все вхождения $x$ в $C$ являются связанными.

Тип $\mathds 1$ является вырожденным примером индуктивного типа, так мы постулируем в нем наличие всего одного
элемента, и на его примере трудно увидеть насколько мощными конструкциями являются индуктивные типы. 
Более иллюстративным примером является тип $\mathbb B$, который населен двумя примитивными константами
$\mathbb T$ и $\mathbb F$.
\begin{multicols}{3}
    \begin{prooftree}
        \AxiomC{$\Gamma \; ctx$}
            \RightLabel{$\mathbb B-FORM$}
        \UnaryInfC{$\Gamma \vdash \mathbb B : \mathcal U_0$}
    \end{prooftree}
    \begin{prooftree}
        \AxiomC{$\Gamma \; ctx$}
            \RightLabel{$\mathbb B - INTRO_1$}
        \UnaryInfC{$\Gamma \vdash \mathbb F : \mathbb B$}
    \end{prooftree}
    \begin{prooftree}
        \AxiomC{$\Gamma \; ctx$}
            \RightLabel{$\mathbb B - INTRO_2$}
        \UnaryInfC{$\Gamma \vdash \mathbb T : \mathbb B$}
    \end{prooftree}
\end{multicols}
\begin{prooftree}
    \AxiomC{$\Gamma, x : \mathbb B \vdash C : \mathcal U_i$}
    \AxiomC{$\Gamma \vdash a : C[\mathbb F/x]$}
    \AxiomC{$\Gamma \vdash b : C[\mathbb T/x]$}
    \AxiomC{$\Gamma \vdash c : \mathbb B$}
        \RightLabel{$\mathbb B - ELIM$}
    \QuaternaryInfC{$\Gamma \vdash \ind_{\mathbb B}(x.C, a, b, c) : C[c/x]$}
\end{prooftree}
\begin{prooftree}
    \AxiomC{$\Gamma, x : \mathbb B \vdash C : \mathcal U_i$}
    \AxiomC{$\Gamma \vdash a : C[\mathbb F/x]$}
    \AxiomC{$\Gamma \vdash b : C[\mathbb T/x]$}
        \RightLabel{$\mathbb B - COMP_1$}
    \TrinaryInfC{$\Gamma \vdash \ind_{\mathbb B}(x.C, a, b, \mathbb F)  \doteq a : C[\mathbb F/x]$}
\end{prooftree}
\begin{prooftree}
    \AxiomC{$\Gamma, x : \mathbb B \vdash C : \mathcal U_i$}
    \AxiomC{$\Gamma \vdash a : C[\mathbb F/x]$}
    \AxiomC{$\Gamma \vdash b : C[\mathbb T/x]$}
        \RightLabel{$\mathbb B - COMP_2$}
    \TrinaryInfC{$\Gamma \vdash \ind_{\mathbb B}(x.C, a, b, \mathbb T)  \doteq b : C[\mathbb T/x]$}
\end{prooftree}

Рассмотрим $\mathbb B$ как булеву алгебру из двух элементов. Покажем, что терм
$$\lambda x : \mathbb B . \ind_{\mathbb B}(y.B, \mathbb T, \mathbb F, x)$$
задает функций отрицания на этой алгебре: применение этого терма к $\mathbb F$ эквивалентно константе $\mathbb T$,
а применение к $\mathbb T$ \dash константе $\mathbb F$. Для начала покажем, что наша
$\lambda$-абстракция имеет ожидаемый тип $\prod_{(x : \mathbb B)} \mathbb B$
(переводит термы типа $\mathbb B$ в термы типа $\mathbb B$):
\begin{prooftree}
    \AxiomC{$\dots$}
    \UnaryInfC{$x : \mathbb B, y : \mathbb B \vdash \mathbb B : \mathcal U_0$}
    \AxiomC{$\dots$}
    \UnaryInfC{$x : \mathbb B \vdash \mathbb T : \mathbb B$}
    \AxiomC{$\dots$}
    \UnaryInfC{$x : \mathbb B \vdash \mathbb F : \mathbb B$}
    \AxiomC{$\dots$}
    \UnaryInfC{$x : \mathbb B \vdash x : \mathbb B$}
        \RightLabel{$\mathbb B - ELIM$}
    \QuaternaryInfC{$x : \mathbb B \vdash \ind_{\mathbb B}(y.\mathbb B, \mathbb T, \mathbb F, x) : \mathbb B$}
    \UnaryInfC{$\vdash (\lambda x : \mathbb B . \ind_{\mathbb B}(y.\mathbb B, \mathbb T, \mathbb F, x)) : \prod \limits_{x : \mathbb B} \mathbb B$}
\end{prooftree}
Теперь покажем, что такая $\lambda$-абстракция действительно вычисляет функцию отрицания.
Действительно, с одной стороны
\begin{prooftree}
    \AxiomC{$\dots$}
    \UnaryInfC{$y : \mathbb B \vdash \mathbb B : \mathcal U_0$}
    \AxiomC{$\dots$}
    \UnaryInfC{$\vdash \mathbb T : \mathbb B$}
    \AxiomC{$\dots$}
    \UnaryInfC{$\vdash \mathbb F : \mathbb B$}
        \RightLabel{$\mathbb B - COMP_1$}
    \TrinaryInfC{$\vdash \ind_{\mathbb B}(y.\mathbb B, \mathbb T, \mathbb F, \mathbb F) \doteq \mathbb T : \mathbb B$}
\end{prooftree}
и
\begin{prooftree}
    \AxiomC{$\dots$}
    \UnaryInfC{$y : \mathbb B \vdash \mathbb B : \mathcal U_0$}
    \AxiomC{$\dots$}
    \UnaryInfC{$\vdash \mathbb T : \mathbb B$}
    \AxiomC{$\dots$}
    \UnaryInfC{$\vdash \mathbb F : \mathbb B$}
        \RightLabel{$\mathbb B - COMP_2$}
    \TrinaryInfC{$\vdash \ind_{\mathbb B}(y.\mathbb B, \mathbb T, \mathbb F, \mathbb T) \doteq \mathbb F : \mathbb B$}
\end{prooftree}
С другой стороны, выводим
\begin{prooftree}
    \AxiomC{$\dots$}
    \UnaryInfC{$x : \mathbb B \vdash \ind_{\mathbb B}(y.\mathbb B, \mathbb T, \mathbb F, x) : \mathbb B$}
    \AxiomC{$\dots$}
    \UnaryInfC{$\vdash \mathbb F : \mathbb B$}
        \RightLabel{$\Pi-COMP$}
    \BinaryInfC{$\vdash (\lambda x : \mathbb B . \ind_{\mathbb B}(y.\mathbb B, \mathbb T, \mathbb F, x))(\mathbb F) \doteq \ind_{\mathbb B}(y.\mathbb B, \mathbb T, \mathbb F, \mathbb F) : \mathbb B$}
\end{prooftree}
и
\begin{prooftree}
    \AxiomC{$\dots$}
    \UnaryInfC{$x : \mathbb B \vdash \ind_{\mathbb B}(y.\mathbb B, \mathbb T, \mathbb F, x) : \mathbb B$}
    \AxiomC{$\dots$}
    \UnaryInfC{$\vdash \mathbb T : \mathbb B$}
        \RightLabel{$\Pi-COMP$}
    \BinaryInfC{$\vdash (\lambda x : \mathbb B . \ind_{\mathbb B}(y.\mathbb B, \mathbb T, \mathbb F, x))(\mathbb T) \doteq \ind_{\mathbb B}(y.\mathbb B, \mathbb T, \mathbb F, \mathbb T) : \mathbb B$}
\end{prooftree}
А значит, пользуясь базовым правилом транзитивности для $\doteq$, мы можем вывести суждения
$$\vdash (\lambda x : \mathbb B . \ind_{\mathbb B}(y.\mathbb B, \mathbb T, \mathbb F, x))(\mathbb F) \doteq \mathbb T : \mathbb B$$ 
и
$$\vdash (\lambda x : \mathbb B . \ind_{\mathbb B}(y.\mathbb B, \mathbb T, \mathbb F, x))(\mathbb T) \doteq \mathbb F : \mathbb B$$ 

Наиболее хорошо, пожалуй, всю вычислительную мощь индукторов можно показать на типе натуральных чисел $\mathbb N$.
\begin{multicols}{2}
    \begin{prooftree}
        \AxiomC{$\Gamma \; ctx$}
            \RightLabel{$\mathbb N - FORM$}
        \UnaryInfC{$\Gamma \vdash \mathbb N : \mathcal U_0$}
    \end{prooftree}
    \begin{prooftree}
        \AxiomC{$\Gamma \; ctx$}
            \RightLabel{$\mathbb N - INTRO_1$}
        \UnaryInfC{$\Gamma \vdash zero : \mathbb N$}
    \end{prooftree}
\end{multicols}
\begin{prooftree}
    \AxiomC{$\Gamma \vdash n : \mathbb N$}
        \RightLabel{$\mathbb N - INTRO_2$}
    \UnaryInfC{$\Gamma \vdash suc(n) : \mathbb N$}
\end{prooftree}
\begin{prooftree}
    \AxiomC{$\Gamma, x : \mathbb N \vdash C : \mathcal U_i$}
    \AxiomC{$\Gamma \vdash a : C[0/x]$}
    \AxiomC{$\Gamma, x : \mathbb N, y : C \vdash b : C[suc(x)/x]$}
    \AxiomC{$\Gamma \vdash n : \mathbb N$}
    \QuaternaryInfC{$\Gamma \vdash \ind_{\mathbb N}(x.C, a, x.y.b, n) : C[n/x]$}
\end{prooftree}
$$(\mathbb N - ELIM)$$
\begin{prooftree}
    \AxiomC{$\Gamma, x : \mathbb N \vdash C : \mathcal U_i$}
    \AxiomC{$\Gamma \vdash a : C[zero/x]$}
    \AxiomC{$\Gamma, x : \mathbb N, y : C \vdash b : C[suc(x)/x]$}
        \RightLabel{$\mathbb N - COMP_1$}
    \TrinaryInfC{$\Gamma \vdash \ind_{\mathbb N}(x.C, a, x.y.b, zero) \doteq a : C[zero/x]$}
\end{prooftree}
\begin{prooftree}
    \AxiomC{$\Gamma, x : \mathbb N \vdash C : \mathcal U_i$}
    \AxiomC{$\Gamma \vdash a : C[zero/x]$}
    \AxiomC{$\Gamma, x : \mathbb N, y : C \vdash b : C[suc(x)/x]$}
    \AxiomC{$\Gamma \vdash n : \mathbb N$}
    \QuaternaryInfC{$\Gamma \vdash \ind_{\mathbb N}(x.C, a, x.y.b, suc(n)) \doteq b[n/x][(\ind_{\mathbb N}(x.C, a, x.y.b, n))/y] : C[suc(n)/x]$}
\end{prooftree}
$$(\mathbb N - COMP_2)$$

Тип $\mathbb N$ населён примитивной константой $zero$, а также снабжён функциональной константой $suc$, которая
позволяет получать последователей $n+1$. Правило $\mathbb N - ELIM$ гласит, что для любого
типа $C$, если можно предъявить терм типа $C(zero)$ и, в предположении что $x$ имеет тип $\mathbb N$, а
$y$ имеет тип $C(x)$, то с помощью индуктора $\ind_{\mathbb N}$ можно получить терм типа $C(n)$ для любого
$n$ из типа $\mathbb N$. Правила $\mathbb N - COMP_1$ и $\mathbb N - COMP_2$ же гласят, что с вычислительной
точки зрения индукторы задают в точности тотальные рекурсивные функции на натуральных числах:
$$f(0) = c$$
$$f(n+1) = g(f(n))$$
В качестве игрушечного примера, покажем, что с помощью $\ind_{\mathbb N}$ можно построить функцию
$n \mapsto n + 1$. Действительно,
\begin{prooftree}
    \AxiomC{$\dots$}
    \UnaryInfC{$x : \mathbb N \vdash \mathbb N : \mathcal U_0$}
    \AxiomC{$\dots$}
    \UnaryInfC{$\vdash suc(zero) : \mathbb N$}
    \AxiomC{$\dots$}
    \UnaryInfC{$x : \mathbb N, y : \mathbb N \vdash suc(y) : \mathbb N$}
    \TrinaryInfC{$\vdash \ind_{\mathbb N}(x.\mathbb N, suc(zero), x.y.suc(y), zero) \doteq suc(zero) : \mathbb N$}
\end{prooftree}
и
\begin{prooftree}
    \AxiomC{$\dots$}
    \UnaryInfC{$x : \mathbb N \vdash \mathbb N : \mathcal U_0$}
    \AxiomC{$\dots$}
    \UnaryInfC{$\vdash suc(zero) : \mathbb N$}
    \AxiomC{$\dots$}
    \UnaryInfC{$x : \mathbb N, y : \mathbb N \vdash suc(y) : \mathbb N$}
    \AxiomC{$\dots$}
    \UnaryInfC{$\vdash zero : \mathbb N$}
    \QuaternaryInfC{$\vdash \ind_{\mathbb N}(x.\mathbb N, suc(zero), x.y.suc(y), suc(zero)) \doteq suc(\ind_{\mathbb N}(x.\mathbb N, suc(zero), x.y.suc(y), zero)) : \mathbb N$}
\end{prooftree}
Объединим два этих вывода с помощью правила $\mathbb N-INTRO_2-EQ$ и выведем суждение
$$\vdash \ind_{\mathbb N}(x.\mathbb N, suc(zero), x.y.suc(y), suc(zero)) \doteq suc(suc(zero)) : \mathbb N$$
Далее, внешней индукцией по длине терма-нумерала $n = \underbrace{suc(suc(\dots suc(}_{n \text{-раз}}zero)\dots)$
мы можем доказать выводимость суждений
$$\vdash \ind_{\mathbb N}(x.\mathbb N, suc(zero), x.y.suc(y), suc(suc(zero))) \doteq suc(suc(suc(zero))) : \mathbb N$$
$$\vdash \ind_{\mathbb N}(x.\mathbb N, suc(zero), x.y.suc(y), suc(suc(suc(zero)))) \doteq suc(suc(suc(suc(zero)))) : \mathbb N$$
$$\vdash \ind_{\mathbb N}(x.\mathbb N, suc(zero), x.y.suc(y), suc(suc(suc(suc(zero))))) \doteq suc(suc(suc(suc(suc(zero))))) : \mathbb N$$
$$\dots$$

Теперь рассмотрим более содержательный пример: сложение натуральных чисел. Для начала заметим, что
\begin{prooftree}
    \AxiomC{$\dots$}
    \UnaryInfC{$n : \mathbb N, m : \mathbb N, x : \mathbb N \vdash \mathbb N : \mathcal U_0$}
    \AxiomC{$\dots$}
    \UnaryInfC{$n : \mathbb N, m : \mathbb N \vdash m : \mathbb N$}
    \alwaysNoLine
    \UnaryInfC{$\dots$}
    \alwaysSingleLine
    \UnaryInfC{$n : \mathbb N, m : \mathbb N \vdash n : \mathbb N$}
    \AxiomC{$\dots$}
    \UnaryInfC{$n : \mathbb N, m : \mathbb N, x : \mathbb N, y : \mathbb N \vdash suc(y) : \mathbb N$}
    \TrinaryInfC{$n : \mathbb N, m : \mathbb m \vdash \ind_{\mathbb N}(x.\mathbb N, n, x.y.suc(y), m) : \mathbb N$}
\end{prooftree}
а значит выводимы суждения
$$\vdash \lambda n : \mathbb N . \lambda m : \mathbb N . \ind_{\mathbb N}(x.\mathbb N, n, x.y.suc(y), m) : \prod_{n : \mathbb N} \prod_{m : \mathbb N} \mathbb N$$
$$\vdash \lambda n : \mathbb N . \lambda m : \mathbb N . \ind_{\mathbb N}(x.\mathbb N, n, x.y.suc(y), m)(l)(0) \doteq l : \mathbb N$$
$$\vdash \lambda n : \mathbb N . \lambda m : \mathbb N . \ind_{\mathbb N}(x.\mathbb N, n, x.y.suc(y), m)(l)(suc(zero)) \doteq suc(l) : \mathbb N$$
$$\vdash \lambda n : \mathbb N . \lambda m : \mathbb N . \ind_{\mathbb N}(x.\mathbb N, n, x.y.suc(y), m)(l)(suc(suc(zero))) \doteq suc(suc(l)) : \mathbb N$$
$$\dots$$
Следовательно, с помощью всё той же внешней индукцией по длине нумералов, для любых натуральных $l$ и $k$, $\lambda$-аппликация
$$\lambda n : \mathbb N . \lambda m : \mathbb N . \ind_{\mathbb N}(x.\mathbb N, n, x.y.suc(y), m)(l)(k)$$
$\doteq$-эквивалентна нумералу для натурального числа $l + k$.

Вернёмся к вопросу логических конструкций с точки зрения интерпретации Брауэра \dash Гейтинга \dash Колмогорова.
Согласно БГК, формуле $\exists x \phi(x)$, выражающей существование некоторого объекта $x$ для которого истинно
свойство $\phi(x)$. С теоретико-типовой точки зрения, для любого зависимого типа $P : \prod_{x : A} B$ 
нам  нужно уметь строить тип пар следующего вида: первым элементом пары является терм $a$ типа $A$, а вторым \dash
терм $b$ типа $B(a)$, который конструктивно доказывает, что для $a$ действительно верно свойство $P(a)$.
В $MLTT$ такой тип пар принято вводить $\Sigma$-абстракцией, называть типом зависимой пары
и описывать следующем набором правил:
\begin{prooftree}
    \AxiomC{$\Gamma \vdash A : \mathcal U_i$}
    \AxiomC{$\Gamma, x : A \vdash  B : \mathcal U_i$}
        \RightLabel{$\Sigma-FORM$}
    \BinaryInfC{$\Gamma \vdash (\sum \limits_{x : A} B) : \mathcal U_i$}
\end{prooftree}
\begin{prooftree}
    \AxiomC{$\Gamma, x : A \vdash B : \mathcal U_i$}
    \AxiomC{$\Gamma \vdash a : A$}
    \AxiomC{$\Gamma \vdash b : B[a/x]$}
        \RightLabel{$\Sigma-INTRO$}
    \TrinaryInfC{$\Gamma \vdash (a, b) : \sum \limits_{x : A} B$}
\end{prooftree}
\begin{prooftree}
    \AxiomC{$\Gamma, z : \sum \limits_{x : A} B \vdash C : \mathcal U_i$}
    \AxiomC{$\Gamma, x : A, y : B \vdash g : C[(x,y)/z]$}
    \AxiomC{$\Gamma \vdash p : \sum \limits_{x : A} B$}
        \RightLabel{$\Sigma-ELIM$}
    \TrinaryInfC{$\Gamma \vdash \ind_{\sum_{(x : A)} B}(z.C, x.y.g, p) : C[p/z]$}
\end{prooftree}
\begin{prooftree}
    \AxiomC{$\Gamma, z : \sum \limits_{x : A} B \vdash C : \mathcal U_i$}
    \AxiomC{$\Gamma, x : A, y : B \vdash g : C[(x,y)/z]$}
    \AxiomC{$\Gamma \vdash a : A$}
    \AxiomC{$\Gamma \vdash b : B[a/x]$}
        \RightLabel{$\Sigma-COMP$}
    \QuaternaryInfC{$\Gamma \vdash (\ind_{\sum_{(x : A)} B}(z.C, x.y.g, (a, b))) \doteq g[a/x][b/y] : C[(a, b)/z]$}
\end{prooftree}

Легко заметить, что если в зависимой сумме $\sum_{(x : A)} B$ тип $B$ не зависит от переменной $x$ как терм, то
$\sum_{(x : A)} B$ является теоретико-типовым аналогом прямого произведения. Естественно, мы также
можем определить тип ко-произведения $A \dotplus B$, который, в некотором смысле, населён дизъюнктивным
объединением термов типа $A$ и типа $B$:
\begin{prooftree}
    \AxiomC{$\Gamma \vdash A : \mathcal U_i$}
    \AxiomC{$\Gamma \vdash B : \mathcal U_i$}
        \RightLabel{$\dotplus-FORM$}
    \BinaryInfC{$\Gamma \vdash (A \dotplus B) : \mathcal U_i$}
\end{prooftree}
\begin{prooftree}
    \AxiomC{$\Gamma \vdash A : \mathcal U_i$}
    \AxiomC{$\Gamma \vdash B : \mathcal U_i$}
    \AxiomC{$\Gamma \vdash a : A$}
        \RightLabel{$\dotplus - INTRO_1$}
    \TrinaryInfC{$\Gamma \vdash inl(a) : A \dotplus B$}
\end{prooftree}
\begin{prooftree}
    \AxiomC{$\Gamma \vdash A : \mathcal U_i$}
    \AxiomC{$\Gamma \vdash B : \mathcal U_i$}
    \AxiomC{$\Gamma \vdash b : B$}
        \RightLabel{$\dotplus - INTRO_2$}
    \TrinaryInfC{$\Gamma \vdash inr(b) : A \dotplus B$}
\end{prooftree}
\begin{prooftree}
    \AxiomC{$\Gamma, x : A \vdash c : C[inl(x)/z]$}
    \alwaysNoLine
    \AxiomC{$\Gamma, z : (A \dotplus B) \vdash C : \mathcal U_i$}
    \UnaryInfC{$\Gamma, y : B \vdash d : C[inr(y)/z]$}
    \alwaysSingleLine
    \AxiomC{$\Gamma \vdash e : (A \dotplus B)$}
        \RightLabel{$\dotplus - ELIM$}
    \TrinaryInfC{$\Gamma \vdash \ind_{A \dotplus B}(z.C, x.c, y.d, e) : C[e/z]$}
\end{prooftree}
\begin{prooftree}
    \AxiomC{$\Gamma, x : A \vdash c : C[inl(x)/z]$}
    \alwaysNoLine
    \AxiomC{$\Gamma, z : (A \dotplus B) \vdash C : \mathcal U_i$}
    \UnaryInfC{$\Gamma, y : B \vdash d : C[inr(y)/z]$}
    \alwaysSingleLine
    \AxiomC{$\Gamma \vdash a : A$}
        \RightLabel{$\dotplus - COMP_1$}
    \TrinaryInfC{$\Gamma \vdash (\ind_{A \dotplus B}(z.C, x.c, y.d, inl(a))) \doteq c[a/x] : C[inl(a)/z]$}
\end{prooftree}
\begin{prooftree}
    \AxiomC{$\Gamma, x : A \vdash c : C[inl(x)/z]$}
    \alwaysNoLine
    \AxiomC{$\Gamma, z : (A \dotplus B) \vdash C : \mathcal U_i$}
    \UnaryInfC{$\Gamma, y : B \vdash d : C[inr(y)/z]$}
    \alwaysSingleLine
    \AxiomC{$\Gamma \vdash b : B$}
        \RightLabel{$\dotplus - COMP_2$}
    \TrinaryInfC{$\Gamma \vdash (\ind_{A \dotplus B}(z.C, x.c, y.d, inr(b))) \doteq d[b/y] : C[inr(b)/z]$}
\end{prooftree}

При построении формальных оснований математики, нам бы естественно хотелось иметь возможность формализовать
и доказывать утверждения о равенствах объектов. Заметим, что в $MLTT$ отношение $\doteq$ на роль равенства
претендовать не может: любое утверждение, в том числе и о равенстве объектов, должно выражаться некоторым типом,
а $\doteq$ задает разбиение термов на вычислительно-эквивалентные классы. Кодировать равенство на языке типов
$MLTT$ мы будем с помощью индуктивного типа $\equiv$:
\begin{prooftree}
    \AxiomC{$\Gamma \vdash A : \mathcal U_i$}
    \AxiomC{$\Gamma \vdash a : A$}
    \AxiomC{$\Gamma \vdash b : A$}
        \RightLabel{$\equiv-FORM$}
    \TrinaryInfC{$\Gamma \vdash (a \equiv_A b) : \mathcal U_i$}
\end{prooftree}
\begin{prooftree}
    \AxiomC{$\Gamma \vdash A : \mathcal U_i$}
    \AxiomC{$\Gamma \vdash a : A$}
        \RightLabel{$\equiv-INTRO$}
    \BinaryInfC{$\Gamma \vdash refl_a : a \equiv_A a$}
\end{prooftree}
\begin{prooftree}
    \AxiomC{$\Gamma, x : A, y : A, p : (x \equiv_A y) \vdash C : \mathcal U_i$}
    \AxiomC{$\Gamma, z : A \vdash c : C[z/x][z/y][refl_z/p]$}
    \alwaysNoLine
    \BinaryInfC{$\Gamma \vdash a : A$}
    \UnaryInfC{$\Gamma \vdash b : A$}
    \UnaryInfC{$\Gamma \vdash p' : a \equiv_A b$}
    \alwaysSingleLine
        \RightLabel{$\equiv-ELIM$}
    \UnaryInfC{$\Gamma \vdash \ind_{\equiv_A}(x.y.p.C, z.c, a, b, p') : C[a/x][b/y][p'/p]$}
\end{prooftree}
\begin{prooftree}
    \AxiomC{$\Gamma, x : A, y : A, p : (x \equiv_A y) \vdash C : \mathcal U_i$}
    \AxiomC{$\Gamma, z : A \vdash c : C[z/x][z/y][refl_z/p]$}
    \alwaysNoLine
    \BinaryInfC{$\Gamma \vdash a : A$}
    \alwaysSingleLine
        \RightLabel{$\equiv-COMP$}
    \UnaryInfC{$\Gamma \vdash (\ind_{\equiv_A}(x.y.p.C, z.c, a, a, refl_a)) \doteq c[a/z] : C[a/x][a/y][refl_a/p]$}
\end{prooftree}
Каждый тип $\equiv_A$ населен константами $refl_a$, которые выражают принцип рефлексивности равенства.
Правило $\equiv-ELIM$ же гласит, что если мы хотим доказать некоторое свойство равных объектов $a$ и $b$,
то нам достаточно рассмотреть случай, когда $a$ и $b$ совпадают синтаксически как слова языка $\lambda$-термов.
Как мы увидим далее, этого достаточно, чтобы $\equiv$ выражал равенство в привычном смысле.
Правило $\equiv-FORM-EQ_1$ утверждает, что термы, равные в смысле $\doteq$, также равны в смысле $\equiv$:
\begin{prooftree}
    \AxiomC{$\Gamma \vdash A : \mathcal U_i$}
    \AxiomC{$\Gamma \vdash a \doteq a' : A$}
    \AxiomC{$\Gamma \vdash b \doteq b' : A$}
        \RightLabel{$\equiv-FORM-EQ$}
    \TrinaryInfC{$\Gamma \vdash (a \equiv_A a') \doteq (b \equiv_A b') : \mathcal U_i$}
\end{prooftree}
А значит, в частности, если $\vdash a \doteq b : A$, то $\vdash refl_a : a \equiv_A b$.

Последним индуктивным типом $MLTT$ является пустой тип $\mathbb 0$:
\begin{multicols}{2}
\begin{prooftree}
    \AxiomC{$\Gamma \; ctx$}
        \RightLabel{$\mathbb 0 - FORM$}
    \UnaryInfC{$\Gamma \vdash \mathbb 0 : \mathcal U_0$}
\end{prooftree}
\begin{prooftree}
    \AxiomC{$\Gamma, x : \mathbb 0 \vdash C : \mathcal U_i$}
    \AxiomC{$\Gamma \vdash a : \mathbb 0$}
        \RightLabel{$\mathbb 0 - ELIM$}
    \BinaryInfC{$\Gamma \vdash \ind_{\mathbb 0}(x.C, a) : C[a/x]$}
\end{prooftree}
\end{multicols}
В отличии от всех предыдущих индуктивных типов, для $\mathbb 0$ мы не вводим правила введения.
Тип $\mathbb 0$ необходим нам для выражения логического принципа "из лжи следует всё, что угодно".

\begin{lemma}
    \label{FalsoLemma}
    Пусть для некоторого контекста $\Gamma$ выводимо суждение $\Gamma \vdash a : \mathbb 0$.
    Тогда для любого терма $B$, если $B$ является типом (выводимо суждение $\Gamma \vdash B : \mathcal U_i$),
    то найдется такой терм $b$ такой, что выводится $\Gamma \vdash b : B$.
\end{lemma}
\begin{proof}
    Почти сразу следует из правила $\mathbb 0 - ELIM$:
    \begin{prooftree}
        \AxiomC{$\Gamma \vdash B : \mathcal U_i$}
        \UnaryInfC{$\Gamma, x : \mathbb 0 \vdash B : \mathcal U_i$}
        \AxiomC{$\Gamma \vdash a : \mathbb 0$}
        \BinaryInfC{$\Gamma \vdash \ind_{\mathbb 0}(x.B, a) : B$}
    \end{prooftree}
\end{proof}

\begin{definition}
    Типизированное $\lambda$-исчисление называется противоречивым, если для любого типа $A$
    найдется такой терм $a$, что выводимо суждение $\vdash a : A$.
\end{definition}

\begin{corollary}
    \label{MlttInconsistencyCriterionCorollary}
    Исчисление $MLTT$ является противоречивым тогда, и только тогда, когда существует терм $a$ такой,
    что выводимо суждение $\vdash a : \mathbb 0$.
\end{corollary}
\begin{proof}
    Импликация слева направо очевидна. Импликация в обратную сторону
    сразу следует из леммы \ref{FalsoLemma} для случая пустого контекста.
\end{proof}

\begin{lemma}
    \label{ContextContractionLemma}
    Пусть $b$ и $B$ \dash некоторые термы, а $x$ \dash переменная, которая не входит свободно в $b$ и $B$. 
    Тогда для любых $\Gamma$ и $\Delta$, таких что в $MLTT$ выводятся $\Gamma \vdash A : \mathcal U_i$ и
    $(\Gamma, \Delta) \; ctx$, суждение $\Gamma, x : A, \Delta \vdash b : B$ выводится тогда и только тогда,
    когда выводится суждение $\Gamma, \Delta \vdash b : B$.
\end{lemma}
\begin{proof}
    Докажем сначала импликацию слева направо. Проведем индукцию по глубине вывода суждения.
    С помощью выводов глубины ноль мы можем получать только суждения о корректности пустого контекста 
    (правило $ctx-EMP$), поэтому в базовом случае доказывать нечего.

    Пусть теперь суждение $\Gamma, x : A, \Delta \vdash b : B$ выводится c помощью дерева нетривиальной глубины.
    Проверим, с помощью какого правило получено суждение в корне вывода.
    Пусть исходное суждение получалось с помощью правила $Subst_1$:
    \begin{prooftree}
        \AxiomC{$\Gamma, x : A \vdash c : C$}
        \AxiomC{$\Gamma, x : A, y : C, \Delta \vdash b : B$}
        \BinaryInfC{$\Gamma, x: A, \Delta[c/y] \vdash b[c/y] : B[c/y]$}
    \end{prooftree}
    По определению $Subst_1$, после подстановки $c$ вместо $y$ никакая свободная переменная не может стать
    связанной, а значит термы $b$, $B$ и $c$ не содержат свободно $x$. А значит мы можем применить
    гипотезу индукции к выводам посылок $\Gamma, x : A \vdash c : C$ и $\Gamma, x : A, y : C, \Delta \vdash b : B$,
    получив тем самым выводимость $\Gamma \vdash c : C$ и $\Gamma y : C, \Delta \vdash b : B$,
    и вывести затем требуемое суждение $\Gamma, \Delta[c/y] \vdash b[c/y] : B[c/y]$.

    Если же исходное суждение выводилось с помощью правила
    $Wkg_1$, то без потери общности мы можем считать, что наше суждение имеет вид $\Gamma, x : A, y : C, \Delta \vdash b : B$, а в посылке мы имеем
    суждение вида $\Gamma, x : A, \Delta \vdash b : B$. Причем термы $b$ и $B$ не изменились в посылке,
    а значит $x$ все еще не входит в них свободно. Применив гипотезу индукции к выводу посылки и совместив
    полученный новый вывод с правилом $Wkg_1$, выводим требуемое суждение $\Gamma, y : C, \Delta \vdash b : B$.

    Наконец, пусть исходное суждение выводится с помощью правил $INTRO$, $ELIM$ или $FORM$. Посылки
    таких правил содержит термы, в которых могут появляться новые свободные переменные. Но каждая такая
    свободная переменная при этом попадает в контекст суждения-посылки. Следовательно, так как контекст не может содержать
    одну и ту же переменную два раза, $x$ также не входит свободно не в один терм из посылок, а значит мы
    вновь можем воспользоваться гипотезой индукции.

    Импликация справа налево тривиальна: получив суждение $\Gamma, \Delta \vdash b : B$, мы можем сразу
    применить правило $Wkg_1$ и вывести $\Gamma, x : A, \Delta \vdash b : B$.
\end{proof}

Аналогичными индуктивными рассуждениями можно доказать аналог леммы \ref{ContextContractionLemma} для
суждений об эквивалентности термов $\doteq$.
\begin{lemma}
    \label{ContextContractionEqLemma}
    Пусть термы $b, b', B$ не содержат свободных вхождений переменной $x$, а тип $A$ с контекстами
    $\Gamma, \Delta$ удовлетворяют условию леммы \ref{ContextContractionLemma}. Тогда суждение
    $\Gamma, x : A, \Delta \vdash b \doteq b' : B$ выводится в $MLTT$ тогда и только тогда, когда
    выводится суждение $\Gamma, \Delta \vdash b \doteq b' : B$.
\end{lemma}

\begin{corollary}
    Пусть $b, b'$ и $B$ \dash замкнутые термы. Тогда для любого контекста $\Gamma$ в $MLTT$
    верно следующие:
    \begin{enumerate}
        \item Выводимость суждения $\Gamma \vdash b : B$ $\iff$ выводимость суждения $\vdash b : B$.
        \item Выводимость суждения $\Gamma \vdash b \doteq b' : B$ $\iff$ выводимость суждения $\vdash b \doteq b' : B$.
    \end{enumerate}
\end{corollary}

\begin{lemma}
    \label{TypingDecidabilityLemma}
    Пусть $t$ \dash произвольный терм, и $\Gamma$ \dash произвольный контекст $MLTT$. Тогда задача выводимости
    суждения о типизации $t$ в контексте $\Gamma$ алгоритмически разрешима. Более того,
    если $t$ типизуем, то алгоритм может явно предъявить некоторый тип $T$ для $t$.
\end{lemma}
\begin{proof}
    Согласно лемме \ref{ContextContractionLemma}, нам достаточно рассматривать случаи, когда в $\Gamma$ состоит
    из свободных переменных терма $t$. Проведем индукцию по длине суммарной длине $l$ терма $t$ и типов из $\Gamma$:
    $$||\Gamma|| = \sum \limits_{(x_i : A_i) \in \Gamma} |A_i|$$
    $$l(t, \Gamma) = |t| + ||\Gamma||$$
    Мы будем разбирать термы согласно их формальной грамматике, параллельно строя дерево вывода суждения о его
    типизации.

    Имеем три базовых случая длины 1: примитивная константа в пустом контексте, функциональная константа в пустом
    контексте и переменная в пустом контексте. В первом случае, наша константа имеет вид $\mathcal U_i$, 
    $\mathbb 1$, $\mathbb 0$, $\mathbb B$, $*$, $\mathbb F$ или $\mathbb  T$, и они все однозначно типизуются в пустом контексте
    выводом глубины один ($ctx_EMP$, а затем соответствующие правило введения или формации). В двух
    оставшихся случаях мы говорим, что терм не типизуем и останавливаем алгоритм (всякая функциональная
    константа не типизуема сама по себе в нашем представлении $MLTT$).

    Рассмотрим теперь случаи нетривиальной длины. Во-первых, случай $t = x$, а $\Gamma = (x : A)$.
    По-правилу $Var$, типизация $x$ возможно только если $A$ типизуем в пустом контексте некоторым $\mathcal U_i$. Но
    $$l(x, (x : A)) = 1 + |A| = 1 + l(A, \emptyset) > l(A, \emptyset)$$
    а значит, по предположению индукции, проблема типизации разрешима для $A$ в пустом контексте, и,
    если $A$ все же типизуем, то нам достаточно будет проверить, что тип $A$ равен $\mathcal U_i$.
    
    Во-вторых, $t$ может иметь вид $\lambda x : A . b$, $\prod_{(x : A) b}$ или $\sum_{(x : A)} b$ и при
    выводе суждения последние правило может быть только $\Pi-INTRO$, $\Pi-FORM$ или $\Sigma-FORM$
    соответственно. При этом
    $$l(\lambda x : A . b, \Gamma) = l(\prod_{(x : A)} b, \Gamma) = l(\sum_{(x : A)} b, \Gamma) \geq$$
    $$\geq 2 + |A| + |b| + ||\Gamma|| > |b| + ||\Gamma, x : A| = l(b, (\Gamma, x : A)) > l(A, \Gamma)$$
    Следовательно, по предположению индукции, мы можем разрешить проблему типизации для посылок
    правил введения и формации.

    Наконец, пусть $t$ имеет вид $a(b)$. Если $a$ является составным термом, то единственное возможное правило
    типизации \dash это $\Pi-ELIM$. При этом
    $$l(a(b), \Gamma) = l(a, \Gamma) + l(b, \Gamma) + 2$$
    и мы вновь можем воспользоваться гипотезой индукции.

    Если же $a$ является функциональной константой $f$, то $f$ однозначно определяет последние возможное
    правило введения, формации, или элиминации. Рассмотрим например случай $\ind_{\mathbb 1}$.
    Тогда последним правилом в выводе суждения о типизации может быть только $\mathbb 1 - ELIM$. Расписав
    сокращение $x.C$, получаем, что в единственный возможный вывод суждения о типизации $f(b)$
    может иметь вид
    \begin{prooftree}
        \AxiomC{$\dots$}
        \UnaryInfC{$\Gamma, x : \mathbb 1 \vdash C : \mathcal U_i$}
        \AxiomC{$\dots$}
        \UnaryInfC{$\Gamma \vdash c : C[*/x]$}
        \AxiomC{$\dots$}
        \UnaryInfC{$\Gamma \vdash b : \mathbb 1$}
        \TrinaryInfC{$\Gamma \vdash \ind_{\mathbb 1}((\lambda x : \mathbb 1 . C), c, b) : C[b/x]$}
    \end{prooftree}
    Сразу становится очевидно, что 
    $$l(\ind_{\mathbb 1}((\lambda x : \mathbb 1 . C), c, b), \Gamma) > l(C, (\Gamma, x : \mathbb 1))$$
    $$l(\ind_{\mathbb 1}((\lambda x : \mathbb 1 . C), c, b), \Gamma) > l(c, \Gamma)$$
    $$l(\ind_{\mathbb 1}((\lambda x : \mathbb 1 . C), c, b), \Gamma) > l(b, \Gamma)$$
    А значит мы можем воспользоваться предположением индукции. Аналогичное рассуждение работает
    и для других функциональных констант.
\end{proof}

\begin{corollary}
    Если в $MLTT$ выводится суждение $\vdash t : T$, то $t$ и $T$ \dash замкнутые термы.
\end{corollary}

Пусть $C_{MLTT}$ \dash множество всех замкнутых типизуемых термов $MLTT$. Определим
формальную систему переписывания термов $(C_{MLTT}, \rightarrow_{COMP}, \twoheadrightarrow_{COMP})$:
термы $a, b \in C_{MLTT}$ находятся в отношение $a \rightarrow_{COMP} b$, если $b$
получается из $a$ единственной заменой подтерма $c$ на терм $c'$ такой, что выводимо суждение 
$\Gamma \vdash c \doteq c' : C$ с помощью какого-либо правила $COMP$ ($\Pi-COMP$, $\mathbb 1 - COMP$ и так далее).
Неформально говоря, $\rightarrow_{COMP}$ является аналогом $\beta$-редукции, получаемого чтением
вычислительных правил спара налево (в частности, отношение $\rightarrow_{COMP}$ не является симметричным
и рефлексивным). Несколько примеров:
$$(\lambda x : \mathcal U_0 . (\lambda y : \mathcal U_0 . y)(x)) \rightarrow_{COMP} \lambda x : \mathcal U_0 . x$$
$$(\lambda x : \mathbb B . \ind_{\mathbb 1}(x.\mathbb B, \mathbb F, *))(\mathbb T) \rightarrow_{COMP} (\lambda x : \mathbb B . \mathbb F)(\mathbb T) \rightarrow_{COMP} \mathbb F$$

Очевидно, что благодаря правилам $ELIM-EQ$, $INTRO-EQ$ и $FORM-EQ$, если замкнутые типизуемые термы находится
в отношении $a \rightarrow_{COMP} b$, то выводимо суждение $\vdash a \doteq b : C$. На самом деле,
с помощью подробной индукции по длине термов и выводов можно доказать следующие сильное утверждение.
\begin{theorem}[\cite{Mart75}, \cite{HoTTBook}]
    \label{MlttStrongNormalization}
    
    \begin{enumerate}
        \item Пусть $a$ и $b$ \dash замкнутые типизуемые термы $MLTT$. Суждение \\
        $\vdash a \doteq b : C$ выводимо тогда и только тогда, когда существует такой $c$ из $C_{MLTT}$,
        что $a \twoheadrightarrow_{COMP} c$ и $b \twoheadrightarrow_{COMP} c$.
        \item Система переписывания термов $(C_{MLTT}, \rightarrow_{COMP}, \twoheadrightarrow_{COMP})$
        сильно нормализуема, причем любая нормальная форма удовлетворяет формальной грамматике
        $$v := k \; | \; \lambda x : v . v \; | \; \prod_{(x : v)} v \; | \; \sum_{(x : v)} v \; | \; f(\overrightarrow{v})$$
        $$k := x \; | \; c \; | k(v) \; | \; f(\overrightarrow{v})(k)$$
    \end{enumerate}
    где $f(\overrightarrow{v})$ \dash частичное применение термов вида $v$ к функциональной константе $f$.
\end{theorem}

\begin{corollary}
    Интуиционистская теория типов Мартин-Лёфа непротиворечива.
\end{corollary}
\begin{proof}
    Как мы уже заметили в следствии \ref{MlttInconsistencyCriterionCorollary}, противоречивость
    $MLTT$ равносильна выводимости суждения $\vdash a : \mathbb 0$.
    Пусть такой терм $a$ существует. По теореме \ref{MlttStrongNormalization},
    мы можем привести $a$ к нормальной форме $a'$ и вывести суждение $\vdash a' : \mathbb 0$.
    Пусть $a'$ \dash наименьший замкнутый терм типа $\mathbb 0$ в нормальной форме.

    Терм $a'$ не может быть переменной, примитивной константой, а также термом вида $\lambda x : A . b$,
    $\prod_{(x : A)} B$ или $\sum_{(x : A) B}$. Остается только рассмотреть случай $a' = f(\overrightarrow{v})(k)$.
    Сразу очевидно, что $f$ не может быть функцией $\dotplus$, $inl, inr$, $\equiv_C$, $suc$, а также конструктором
    пары зависимой суммы. $f$  также не может быть индуктором $\ind_{\mathbb 1}$, $\ind_{\mathbb B}$,
    $\ind_{\mathbb N}$, $\ind_{A \dotplus B}$, $\ind_{\Sigma_{(x : A)} B}$ и $\ind_{\equiv_A}$, так
    как замкнутые типизуемые термы с такими функциональными константами всегда будут 
    преобразованы согласно правилам $COMP$.

    Остается только случай $a' = \ind_{\mathbb 0}(x.\mathbb 0, b)$. Но тогда $b$ также является замкнутым
    термом в нормальной форме, и, согласно правилу $\mathbb O-ELIM$, $\vdash b : \mathbb 0$. Но длинна
    терма $b$ очевидно меньше длины $a'$, что противоречит минимальности $a'$.
\end{proof}

Исчисление $MLTT$ непротиворечиво, а значит может служить в качестве формальных оснований для
математики. Однако мы знаем много доказуемо непротиворечивых формальных дедуктивных систем, $MLTT$
не является в этом смысле редким исключением. Что действительно выделяет интуиционистскую теорию типов
Мартин-Лёфа, так эта её "дружелюбность"\ одновременно и к людям, и к компьютерам.

\begin{definition}
    Пусть $(A, \rightarrow, \twoheadrightarrow)$ \dash формальная система преобразований такая, что
    для любых $a, b, b' \in A$, если $a \rightarrow b$ и $b \rightarrow b'$, то найдется такой $c \in A$,
    что $b \twoheadrightarrow c$ и $b' \twoheadrightarrow c$.
    $$
    \begin{tikzcd}
        & \arrow[ld] a \arrow[rd] & \\
        b \arrow[rd, two heads, dashed] & & \arrow[ld, two heads, dashed] b' \\
        & c &
    \end{tikzcd}
    $$
    Тогда говорят, что $(A, \rightarrow, \twoheadrightarrow)$ удовлетворяет слабому свойству Чёрча \dash Россера.
\end{definition}

\begin{lemma}[Ньюман]
    \label{NewmanLemma}
    Пусть формальная система преобразований $(A, \rightarrow, \twoheadrightarrow)$ \\ сильно-нормализуема
    и удовлетворяет слабому свойству Чёрча \dash Россера. Тогда $(A, \rightarrow, \twoheadrightarrow)$
    также удовлетворяет обычному свойству Чёрча \dash Россера.
\end{lemma}
\begin{proof}
    Для начала заметим, что сильно-нормализуемая формальная система преобразований обладает свойством
    Чёрча\dash Россера тогда и только тогда, когда нормальная форма каждого терма единственна.
    Действительно, импликация слева направо является частным случаем леммы \ref{UniqueNormalFormLemma}.
    Обратно, пусть $a \twoheadrightarrow b$ и $a \twoheadrightarrow b'$. Если $c$ \dash единственная
    нормальная форма $a$, то $b \twoheadrightarrow c$ и $b' \twoheadrightarrow c$ в силу сильной
    нормализуемости. Что и требовалось.

    Пусть теперь наша система преобразований сильно-нормализуема, обладает слабым свойством Чёрча \dash Россера,
    но не удовлетворяет обычному свойству Чёрча \dash Россера. Тогда у нас есть минимум один
    элемент $a$ такой, что $a$ редуцируется к двум различным нормальным формам $b_1$ и $b_2$.
    Очевидно, что $a$ сам не может находится в нормальной форме, а значит найдутся такие $a_1$ и $a_2$ такие,
    что 
    $$
    \begin{tikzcd}
        & \arrow[ld] a \arrow[rd] & \\
        a_1 \arrow[d, two heads] & & \arrow[d, two heads] a_2 \\
        b_1 &  & b_2
    \end{tikzcd}
    $$
    По слабому свойству Чёрча \dash Россера найдется $b_3$ такой, что $a_1 \twoheadrightarrow b_3$ и
    $a_2 \twoheadrightarrow b_3$, а в силу сильной нормализуемости можем считать, что $b_3$ находится в
    нормальной форме. Без потери общности можем считать, что $b_1 \neq b_3$, и мы можем повторить вычисление
    для $a_1$. Но тогда мы можем построить бесконечную последовательность
    $$a \rightarrow a_1 \rightarrow a'_1 \rightarrow \dots$$
    что противоречит сильной нормализуемости.
\end{proof}

\begin{corollary}
    Пусть $a$ и $A$ \dash замкнутые термы $MLTT$. Тогда задача существования вывода суждения
    $$\vdash a : A$$
    алгоритмически разрешима. 
\end{corollary}
\begin{proof}
    Для начала, пользуясь алгоритмом из леммы \ref{TypingDecidabilityLemma}, проверим, что $a$ является
    типизуемым, и в случае успеха получим выводимость суждения $$\vdash a : A'$$
    для некоторого замкнутого терма $A'$.

    Заметим, что для $(C_{MLTT}, \rightarrow_{COMP}, \twoheadrightarrow_{COMP})$ выполняется
    слабое свойство Чёрча \dash Россера. Действительно, с помощью $\rightarrow_{COMP}$ можно
    переписать лишь один подтерм за раз. Теорема \ref{StrongNormalizationDefinition} гласит,
    что $(C_{MLTT}, \rightarrow_{COMP}, \twoheadrightarrow_{COMP})$ сильно-нормализуема, а значит,
    по лемме Ньюмана, $(C_{MLTT}, \rightarrow_{COMP}, \twoheadrightarrow_{COMP})$ обладает полным
    свойством Чёрча \dash Россера, и, в силу леммы \ref{UniqueNormalFormLemma}, нормальная форма замкнутого
    типизуемого терма единственна. В частности, мы можем усилить первый пункт теоремы \ref{MlttStrongNormalization}:
    суждение $\vdash a \doteq b : C$ выводится в $MLTT$ тогда и только тогда, когда $a$ и $b$
    редуцируются к одной и той же нормальной форме.

    Приведем термы $A$ и $A'$ к нормальной форме. Порядок редукций не важен, так как замкнутые термы
    сильно нормализуемые, и за конечное число шагов мы всегда придем к нормальной форме. Если
    нормальные форма $A$ и $A'$ равны как слова языка $\lambda$-термов, то суждение
    $\vdash a : A$ выводимо. Если мы пришли к разным нормальным формам, то суждение не выводимо.
\end{proof}

Таким образом, на основе $MLTT$ можно сделать интерактивную систему автоматической проверки доказательств:
мы можем формулировать математические утверждения как типы, после чего писать терм соответствующий данному
типу и просить компьютерный алгоритм проверить терм и тип на соответствие. Если алгоритм выдал
положительные результат, мы можем быть уверены, что наше утверждение доказуемо в нашей формальной системе,
причем явно писать громоздкие деревья вывода нам больше не нужно.

\section{Язык Agda 2}

Agda 2 (далее просто Agda) \dash это автоматическая система проверки доказательств и язык программирования общего назначения.
Теория типов, реализуемая Agda \dash это расширенный и модифицированный вариант интуиционистской теории типов 
Мартин-Лёфа. С синтаксической точки зрения, Agda испытала сильное влияние языка Haskell, и стандартный
код на Agda имеет следующий вид: мы объявляем именную константу и постулируем её тип с помощью записи
$$ name \; : \; type$$
а затем присваиваем именной константе конкретный $\lambda$-терм с помощью записи
$$ name = term $$
Если тип терма, присвоенного именной константе, совпадает с типом, постулируемым для той же переменной,
то компилятор Agda сообщит об успешной проверки типов, тем самым произведя автоматическую проверку корректности
доказательств. В противном случае, компилятор сообщит об ошибке типизации и выделит подтерм, на котором произошло
расхождение.

С точки зрения обобщенного соответствия Карри \dash Ховарда, объявляя именную константу и постулируя
её тип, мы формулируем математическое утверждение, а присваивая константе $\lambda$-терм,
мы приводим конструктивное доказательство данного математического утверждения. Причем мы можем
использовать именные константы как синонимы для соответствующих термов для написания
других термов в дальнейшем, тем самым делая очень громоздкие термы читаемыми и компактными.
С точки зрения абстрактной теории типов, конструкциям именных констант советует добавление вычислительных 
правил вида
\begin{prooftree}
    \AxiomC{$\Gamma \vdash t : T$}
    \UnaryInfC{$\Gamma \vdash name \doteq t : T$}
\end{prooftree}
для каждой константы $name$ в коде.

Далее мы будем формализовать математику на языке Agda 2, непосредственно приводя вставки кода, 
которые были заранее проверенны на корректность компилятором Agda (версии 2.6.3) на этапе сборки этого документа.

\subsection{Основные конструкции}

Как и во многих других языках программирования, написания кода на Agda начинается с прелюдии:
\AgdaTarget{Level}
\AgdaTarget{lsuc}
\AgdaTarget{lzero}
\AgdaTarget{\_⊔\_}
\begin{code}
{-# OPTIONS --cubical #-} -- Передем параметры проверки типов 
-- и компиляции напрямую из файла с кодом.
-- Единисвенный параметр здесь - это поддержка кубической теории типов
-- (о ней позднее)

open import Agda.Primitive using (Level; lzero; lsuc; _⊔_)
                           renaming (Set to 𝒰) -- Импортируем
-- примтивные тип универсумов и переименовываем их в 𝒰
-- (По некоторым историческим причинам универсумы типов в Agda
-- по умолчанию называются Set. Мы будем придерживаться 
-- нотации из предыдущего раздела.)

module MasterThesis where -- объявляем главный модуль
\end{code}

"Из коробки" \; Agda, по мимо универсумов типов, поддерживает только один примитивный тип: зависимое
произведение. Вот так, например, мы можем записать терм типа 
$$\prod_{(A : \mathcal U_0)} \prod_{(x : A)} A$$
который действует как тождественная функция на любом типе из $\mathcal U_0$:
\begin{code}
id₀ : (A : 𝒰₀) → (x : A) → A
id₀ = λ (A : 𝒰₀) → (λ (x : A) → x)
\end{code}
Причем мы можем не проставлять типы переменных в $\lambda$-абстракции явно, а также не писать зависимую
переменную под знаком зависимого произведения, если результирующий тип от неё не зависит:
\begin{code}
id₀' : (A : 𝒰₀) → A → A
id₀' = λ A x → x
\end{code}
Более того, мы можем не писать $\lambda$-абстракции совсем и использовать специальный сокращенный
синтаксис:
\begin{code}
id₀'' : (A : 𝒰₀) → A → A
id₀'' A x = x
\end{code}

В отличии от оригинальной $MLTT$, теория типов Agda обладает более гибкой и выразительной системой
уровней универсумов: мы постулируем наличие типа уровней \AgdaDatatype{Level}, 
нулевого уровня \AgdaFunction{lzero}, функции взятия следующего уровня \AgdaFunction{lsuc}
и функции взятия максимума \AgdaFunction{\_⊔\_}. На уровне правил о типизации, данные конструкции
описываются следующим образом:
\begin{multicols}{2}
    \begin{prooftree}
        \AxiomC{$\Gamma \; ctx$}
        \UnaryInfC{$\Gamma \vdash Level : \mathcal U_0$}
    \end{prooftree}
    \begin{prooftree}
        \AxiomC{$\Gamma \; ctx$}
        \UnaryInfC{$\Gamma \vdash lzero : Level$}
    \end{prooftree}
\end{multicols}
\begin{multicols}{2}
    \begin{prooftree}
        \AxiomC{$\Gamma \vdash i : Level$}
        \UnaryInfC{$\Gamma \vdash lsuc(i) : Level$}
    \end{prooftree}
    \begin{prooftree}
        \AxiomC{$\Gamma \; ctx$}
        \UnaryInfC{$\Gamma \vdash \mathcal U(lzero) \doteq \mathcal U_0 : \mathcal U_{1}$}
    \end{prooftree}
\end{multicols}
\begin{multicols}{2}
    \begin{prooftree}
        \AxiomC{$\Gamma \vdash i : Level$}
        \UnaryInfC{$\Gamma \vdash \mathcal U(i) \doteq \mathcal U_i : \mathcal U_{i+1}$}
    \end{prooftree}
    \begin{prooftree}
        \AxiomC{$\Gamma \; ctx$}
        \UnaryInfC{$\Gamma \vdash \mathcal U(lsuc(i)) \doteq \mathcal U_{i+1} : \mathcal U_{i + 2}$}
    \end{prooftree}
\end{multicols}
\begin{prooftree}
    \AxiomC{$\Gamma \vdash i : Level$}
    \AxiomC{$\Gamma \vdash l : Level$}
    \BinaryInfC{$\Gamma \vdash i \sqcup l : Level$}
\end{prooftree}
\begin{prooftree}
    \AxiomC{$\Gamma \vdash i : Level$}
    \AxiomC{$\Gamma \vdash l : Level$}
    \BinaryInfC{$\Gamma \vdash i \sqcup l \doteq l \sqcup i : Level$}
\end{prooftree}
\begin{prooftree}
    \AxiomC{$\Gamma \vdash i : Level$}
    \AxiomC{$\Gamma \vdash l : Level$}
    \AxiomC{$i \leq l$}
    \TrinaryInfC{$\Gamma \vdash \mathcal U(i \sqcup l) \doteq \mathcal U_l : \mathcal U_{l + 1}$}
\end{prooftree}

Такое расширение системы уровней универсумов типов позволяет предъявить терм типа
$$\prod_{i : Level} \prod_{A : \mathcal U(i)} \prod_{x : A} A$$
являющемся тождественной функций на термах любого типа из любого универсума:
\AgdaTarget{id}
\begin{code}
id : {i : Level} → {A : 𝒰 i} → A → A
id x = x
\end{code}
Здесь в фигурных скобках мы пишем неявные аргументы, которые нам не обязательно писать эксплицитно,
так как компилятор Agda почти всегда может однозначно восстановить недостающие аргументы. Действительно,
в корректном контексте, типизирую терм $id(t)$, мы однозначно знаем тип $A$ терма $t$, а значит и минимальный
универсум $\mathcal U(i)$, в котором обитает $A$.

Если мы же мы хотим рассуждать строго в $MLTT$, то можем считать, что \AgdaFunction{id} задаёт
не один $\lambda$-терм, а целое семейство: по одному для каждого универсума $\mathcal U_i$.

Теперь определим в Agda базовые индуктивные типы. Каждый индуктивный тип определяется
своим именем и возможными зависимостями (правила $FORM$), а также описанием населяющих
этот тип термов (правила $INTRO$). После таких определений индуктивных типов, Agda 
автоматически выводит соответствующие правила $ELIM$ и $COMP$.
\AgdaTarget{≡}
\AgdaTarget{refl}
\AgdaTarget{𝔹}
\begin{code}
data 𝟙 : 𝒰 where
    * : 𝟙

data 𝔹 : 𝒰 where
    𝔽 : 𝔹
    𝕋 : 𝔹

data 𝟘 : 𝒰 where -- пустой тип

data ℕ : 𝒰 where
    zero : ℕ
    suc : ℕ → ℕ

{-# BUILTIN NATURAL ℕ #-} -- регистрируем ℕ как
-- тип натуральных числе по-умолчанию
-- (в частности, позволяет использовать арабские цифры
-- для записи термов)

data Σ {i j : Level} (A : 𝒰 i) (B : A → 𝒰 j) : 𝒰 (i ⊔ j) where
    _,_ : (x : A) → B x → Σ A B -- специальный синтаксис для инфиксных
-- операторов

infixr 4 _,_ -- устанавливаем приоритет инфиксного 
-- и право-ассоциативность оператора

data _∔_ {i j : Level} (A : 𝒰 i) (B : 𝒰 j) : 𝒰 (i ⊔ j) where
    inl : A → (A ∔ B)
    inr : B → (A ∔ B)

data _≡_ {l : Level} {A : 𝒰 l} : A → A → 𝒰 l where
    refl : (a : A) → a ≡ a

{-# BUILTIN EQUALITY _≡_  #-}

infixl 0 _≡_ -- лево-ассоциативность ≡ как инфиксного оператора
\end{code}

В Agda мы не работаем напрямую с индукторами, вместо них мы определяем эквивалентные им
термы с помощью так называемого сопоставления с образцом (также известного как разбор случаев). 
Так, терм, эквивалентный абстракции
$$(\lambda x : \mathbb B . \ind_{\mathbb B}(y.\mathbb B, \mathbb T, \mathbb F, x)) : \prod \limits_{x : \mathbb B} \mathbb B$$
выражающей функцию отрицания на типе $\mathbb B$, в Agda можно записать следующим образом:
\begin{code}
not : 𝔹 → 𝔹
not 𝔽 = 𝕋
not 𝕋 = 𝔽
\end{code}
Аналогично, можем с помощью разбора случаев по второму аргументу определить функцию сложения на натуральных числах, эквивалентную
терму
$$\lambda n : \mathbb N . \lambda m : \mathbb N . \ind_{\mathbb N}(x.\mathbb N, n, x.y.suc(y), m) : \prod_{n : \mathbb N} \prod_{m : \mathbb N} \mathbb N$$
\begin{code}
_+ᴺ_ : ℕ → ℕ → ℕ
x +ᴺ zero = x
x +ᴺ (suc y) = suc (x +ᴺ y)

infixl 4 _+ᴺ_
\end{code}
(Теперь сразу видно, что сложение на типе натуральных чисел является полным аналогом сложения
на ординалах в теории множеств.)

Сразу же определим и умножение
\begin{code}
_·ᴺ_ : ℕ → ℕ → ℕ
x ·ᴺ zero = zero
x ·ᴺ (suc y) = (x ·ᴺ y) +ᴺ x

infixl 5 _·ᴺ_ -- умножение лево-ассоциативно и имеет приоритет перед сложением
\end{code}

С помощью типа равенства \AgdaDatatype{≡} мы теперь можем показать, что наши функции выше
корректно определенны на конкретных термах. Причем перед проверкой типов Agda приводит все термы к нормальной
форме, в том числе честно вычисляя функции отрицания, сложения и умножения, а значит
для доказательства корректности нам будет достаточно просто применить функциональную константу
(конструктор термов) \AgdaInductiveConstructor{refl} к терму-результату вычисления:
\begin{code}
notProp₁ : not 𝔽 ≡ 𝕋
notProp₁ = refl 𝕋

notProp₂ : not 𝕋 ≡ 𝔽
notProp₂ = refl 𝔽

addProp₁ : ((suc zero) +ᴺ (suc (suc zero))) ≡ (suc (suc (suc zero)))
addProp₁ = refl (suc (suc (suc zero)))

addProp₂ : (3 +ᴺ 2) ≡ 5 -- пользуемся встроенной поддержкой арабских цифр
addProp₂ = refl (suc (suc (suc (suc (suc zero)))))

addProp₃ : (0 +ᴺ (10 +ᴺ 15)) ≡ 25
addProp₃ = refl 25

multProp₁ : (12 ·ᴺ 6) ≡ 72
multProp₁ = refl 72

multProp₂ : 0 ≡ (0 ·ᴺ 1001)
multProp₂ = refl zero
\end{code}

Как мы уже могли понять, все правила типизации термов существуют в некотором внутреннем представлении компилятора
Agda, и мы не можем с ними взаимодействовать напрямую. Это порождает некоторые вопросы. Например,
действительно ли правила элиминации, сгенерированное для нашего индуктивного типа, совпадают с соответствующими
правилами $MLTT$. Тем не менее, мы можем сформулировать всякое правило элиминации как некоторый замкнутый
терм типа зависимого произведения. Так, правило $\mathbb 1-ELIM$ в $MLTT$ эквивалентно
постулированию существования терма 
$$\mathbb 1-ELIM : \prod_{(C : \prod_{(x : \mathbb 1)} \mathcal U_i)} \prod_{(c : C(*))} \prod_{(a : \mathbb 1)} C(a)$$
Давайте явно предъявим термы-элиминаторы для каждого базового индуктивного типа, и тип самым
убедимся, что любое суждение о типизации из $MLTT$ можно перенести в Agda:
\AgdaTarget{ℕ-ELIM}
\begin{code}
𝟙-ELIM : {i : Level} → (C : 𝟙 → 𝒰 i) → (c : C *) → (a : 𝟙) → C a
𝟙-ELIM C c * = c

𝔹-ELIM : {i : Level} → (C : 𝔹 → 𝒰 i) 
         → (a : C 𝔽) 
         → (b : C 𝕋) 
         → (c : 𝔹) → C c
𝔹-ELIM C a b 𝔽 = a
𝔹-ELIM C a b 𝕋 = b

𝟘-ELIM : {i : Level} → (C : 𝟘 → 𝒰 i) → (x : 𝟘) → C x
𝟘-ELIM P = λ () -- специальный паттерн "абсурд" для пустых типов

ℕ-ELIM : {i : Level} → (C : ℕ → 𝒰 i)
         → (a : C 0) -- база структурной индукции
         → ((x : ℕ) → (y : C x) → C (suc x)) -- шаг структурной индукции
         → (n : ℕ) → C n
ℕ-ELIM C base step zero = base
ℕ-ELIM C base step (suc x) = step x IH
    where -- открываем локальное лексическое окружение (локальные термы)
        IH : C x -- аналог предположения индукции
        IH = ℕ-ELIM C base step x

Σ-ELIM : {i j k : Level} → {A : 𝒰 i} → {B : A → 𝒰 j}
         → (C : Σ A B → 𝒰 k)
         → ((x : A) → (y : B x) → C (x , y))
         → (p : Σ A B) → C p
Σ-ELIM C step (a , b) = step a b

∔-ELIM : {i j k : Level} → {A : 𝒰 i} → {B : 𝒰 j}
         → (C : (A ∔ B) → 𝒰 k)
         → ((x : A) → C (inl x))
         → ((y : B) → C (inr y))
         → (e : (A ∔ B)) → C e
∔-ELIM C step₁ step₂ (inl a) = step₁ a
∔-ELIM C step₁ step₂ (inr b) = step₂ b

≡-ELIM : {i j : Level} → {A : 𝒰 i}
         → (C : (x : A) → (y : A) → (p : x ≡ y) → 𝒰 j)
         → ((z : A) → C z z (refl z))
         → (a : A) → (b : A) → (p' : a ≡ b)
         → C a b p'
≡-ELIM C step a a (refl a) = step a
\end{code}
Теперь очевидно, особенно на примере \AgdaFunction{ℕ-ELIM}, что правила элиминации являются
ничем иным, как принципами индукции для индуктивных типов (отсюда исторически и происходит их название).

Естественно, термы-элиминаторы ведут себя как самые настоящие индукторы в вычислительном
смысле. Покажем это на примере типа \AgdaDatatype{𝔹}:
\begin{code}
𝔹-COMP₁ : {i : Level} → (C : 𝔹 → 𝒰 i)
          → (a : C 𝔽)
          → (b : C 𝕋)
          → (𝔹-ELIM C a b 𝔽 ≡ a)
𝔹-COMP₁ C a b = refl a

𝔹-COMP₂ : {i : Level} → (C : 𝔹 → 𝒰 i)
          → (a : C 𝔽)
          → (b : C 𝕋)
          → (𝔹-ELIM C a b 𝕋 ≡ b)
𝔹-COMP₂ C a b = refl b
\end{code}

Перейдём к вопросам формализации и приложений обобщённого соответствия Карри \dash Ховарда.
Для начала покажем, что для типов (если мы воспринимаем их как утверждения) выполнены аналоги
всех естественных интуиционистских логических законов. Начнем с определения
типа прямого произведения (частный случай типа зависимой суммы, как мы уже обсуждали в предыдущей главе),
который будет выступать в роли конъюнкции:
\begin{code}
_×_ : {i j : Level} → 𝒰 i → 𝒰 j → 𝒰 (i ⊔ j)
A × B = Σ A (λ x → B) -- второй аргумент Σ - константная функция,
-- которая всегда редуцируется к терму-типу B

infixl 2 _×_
\end{code}
Также можем определить интуиционистское отрицание на типах: предположение, что тип $A$ населён
ведёт к абсурду
\AgdaTarget{¬}
\begin{code}
¬ : {i : Level} → 𝒰 i → 𝒰 i
¬ A = (A → 𝟘)
\end{code}
Теперь мы можем вывести теоретико-типовые аналоги аксиом интуиционистской логики высказываний.
\begin{code}
Ax1 : {i j : Level} → {A : 𝒰 i} → {B : 𝒰 i}
      → A → B → A
Ax1 = λ a b → a

Ax2 : {i j k : Level} → {A : 𝒰 i} → {B : 𝒰 j} → {C : 𝒰 k}
      → (A → (B → C)) → ((A → B) → (A → C))
Ax2 abc ab = λ x → abc x (ab x)

Ax3 : {i j : Level} → {A : 𝒰 i} → {B : 𝒰 j}
      → (A × B) → A
Ax3 (a , b) = a

Ax4 : {i j : Level} → {A : 𝒰 i} → {B : 𝒰 j}
      → (A × B) → B
Ax4 (a , b) = b

Ax5 : {i j : Level} → {A : 𝒰 i} → {B : 𝒰 j}
      → (A → B → (A × B))
Ax5 a b = (a , b)

Ax6 : {i j : Level} → {A : 𝒰 i} → {B : 𝒰 j}
      → (A → (A ∔ B))
Ax6 a = inl a

Ax7 : {i j : Level} → {A : 𝒰 i} → {B : 𝒰 j}
      → (B → (A ∔ B))
Ax7 b = inr b

Ax9 : {i j k : Level} → {A : 𝒰 i} → {B : 𝒰 j} → {C : 𝒰 k}
      → (A → C) → (B → C) → ((A ∔ B) → C)
Ax9 ac bc (inl a) = ac a
Ax9 ac bc (inr b) = bc b

Ax10 : {i j : Level} → {A : 𝒰 i} → {B : 𝒰 j}
       → (A → B) → (A → ¬ B) → ¬ A
Ax10 {i} {j} {A} {B} ab anb = λ a → anb a (ab a)

Ax11 : {i j : Level} → {A : 𝒰 i} → {B : 𝒰 j}
       → (A → ¬ A → B)
Ax11 {i} {j} {A} {B} a nega = 𝟘-ELIM (λ x → B) (nega a)

MP : {i j : Level} → {A : 𝒰 i} → {B : 𝒰 j} -- правило MP как Π-тип
     → A → (A → B) → B
MP a ab = ab a
\end{code}

Использую комбинаторы выше, мы можем выводить аналоги пропозициональных теорем потчи что
в смысле гильбертовского исчисления
\begin{code}
id' : {i : Level} → {A : 𝒰 i} → A → A
id' {i} {A} = step5
    where
        step1 : A → (A → A) → A
        step1 = Ax1 {i} {i} {A} {A → A}
        step2 : (A → (A → A) → A) → ((A → (A → A)) → (A → A))
        step2 = Ax2 {i} {i} {i} {A} {A → A} {A}
        step3 : (A → (A → A)) → (A → A)
        step3 = MP {i} {i} step1 step2
        step4 : A → (A → A)
        step4 = Ax1 {i} {i} {A} {A}
        step5 : A → A
        step5 = MP {i} {i} step4 step3

id'-prop : {i : Level} → {A : 𝒰 i}
           → (x : A) → (id' x ≡ x)
id'-prop x = refl x
\end{code}
Однако очевидно, что на практике особого смысла в этом нет, и гораздо продуктивнее
выводить соответствующие термы, использую сопоставление с образцом напрямую.

Аксиоме исключенного третьего соответствует тип
\AgdaTarget{LEM}
\begin{code}
LEM : {i : Level} → 𝒰 i → 𝒰 i
LEM A = A ∔ ¬ A
\end{code}
Теоретико-типовой закон исключенного третьего гласит, что по любому типу $A$, мы можем
либо предъявить терм из $A$, либо получить доказательство, что $A$ пуст. Естественно,
это очень сильное свойство и вывести его в такой общей форме невозможно. Однако, для наших
базовых индуктивных типов мы можем вывести частный случай закон исключенного третьего.
\begin{code}
LEM-𝟙 : LEM 𝟙
LEM-𝟙 = inl *

LEM-𝔹 : LEM 𝔹
LEM-𝔹 = inl 𝔽

LEM-𝟘 : LEM 𝟘
LEM-𝟘 = inr (λ x → x)
\end{code}
Типы, для которых доказуем закон исключенного третьего, принято называть разрешимыми.

Принимая \AgdaFunction{LEM} в качестве предпосылки, мы можем, например, доказать
закон снятия двойного отрицания:
\begin{code}
doubleNegation : {i : Level} → {A : 𝒰 i} → LEM A → (¬ (¬ A) → A)
doubleNegation (inl a) doubleNeg = a
doubleNegation {i} {A} (inr nota) doubleNeg = 𝟘-ELIM (λ x → A) (doubleNeg nota)
\end{code}
Сразу заметим, что, как и следовало ожидать от интуиционистской теории, закон
снятия тройного отрицания доказывается без закона исключенного третьего.
\begin{code}
tripleNegation : {i : Level} → {A : 𝒰 i} → ¬ (¬ (¬ A)) → ¬ A
tripleNegation {i} {A} tripleNeg = λ (x : A)  → tripleNeg (λ (y : ¬ A) → y x)
\end{code} 

\begin{thebibliography}{}
    \bibitem[SorUrz06]{SorUrz06}
    M. H. Sorensen, P. Urzycyn. \textit{Lectures on the Curry-Howard \\ Isomorphism.}
    Studies in Logic and the Foundations of Mathematics. \\ Volume 149. 2006. ISBN
    978-0-444-52077-7.

    \bibitem[Gira71]{Gira71}
    J. Girard. \textit{Une Extension De ĽInterpretation De Gödel a ĽAnalyse, \\ Et Son Application  a ĽElimination Des Coupures Dans ĽAnalyse  \\ Et La Theorie Des Types.}
    Studies in Logic and the Foundations of Mathematics. Volume 63. 1971. ISBN 978-0-720-42259-7.

    \bibitem[Zakr07]{Zakr07}
    M. Zakrzewski. \textit{Definable functions in the simply typed lambda-calculus.} \\ 2007.
    arXiv:cs/0701022
    
    \bibitem[Mart75]{Mart75}
    P. Martin-Löf. \textit{An Intuitionistic Theory of Types: Predicative Part.} Studies in Logic and the Foundations of Mathematics,
    Volume 80. 1975. ISBN 978-0-444-10642-1.

    \bibitem[HoTTBook]{HoTTBook}
    S. Awodey et al. \textit{Homotopy Type Theory: Univalent Foundations of Mathematics.} Institute for Advanced Study.
    Version: first-edition-1357-gde0b8e22.
\end{thebibliography}

\end{document} 